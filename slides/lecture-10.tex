\newcommand{\h}{%
handout,%
}

\documentclass[
\h%
ucs,ignorenonframetext,hyperref={pdftex,unicode},xcolor=dvipsnames]{beamer}
\usepackage[utf8x]{inputenc}
\usepackage[russian]{babel}
\usepackage[T2A]{fontenc}
\usepackage{tikz}
\usetikzlibrary{shapes,arrows,positioning,chains,calc}
\usepackage{graphicx}
\usepackage{fixltx2e}
\usepackage{paratype}
\usepackage{booktabs}
\usepackage{xcolor}
\usepackage{pbox}

\newcommand{\screenshot}[1]{
\begin{center}
\includegraphics[width=12cm,keepaspectratio]{./images/#1}
\end{center}
}

\newcommand{\screenshotw}[2]{
\begin{center}
\includegraphics[width=#1,keepaspectratio]{./images/#2}
\end{center}
}

\usecolortheme{crane}
\useoutertheme{infolines}
\setbeamertemplate{navigation symbols}{}

\author[А. М. Пеленицын]{А.~М.~Пеленицын\texorpdfstring{\\}{ }
apel@sfedu.ru}

\date{Весна 2016}

\institute[Мехмат ЮФУ]{Южный федеральный университет \texorpdfstring{\\}{ }
Институт математики, механики и компьютерных наук им. И.\,И.~Воровича\texorpdfstring{\\}{ }
Кафедра информатики и вычислительного эксперимента}

\subtitle{}

\AtBeginSection[]
{
\begin{frame}<beamer>
\frametitle{Содержание}
\tableofcontents[currentsection,hideothersubsections]
\end{frame}
}

\AtBeginSubsection[]
{
\begin{frame}<beamer>
\frametitle{Содержание}
\tableofcontents[currentsection,subsectionstyle=show/shaded/hide]
\end{frame}
}

\newcommand{\nspace}{\hspace{0pt}}
\newcommand{\nbdash}{\nobreakdash-\nspace}
\newcommand{\up}{\textsuperscript}
\newcommand{\bslash}{\textbackslash}

\newcommand{\link}[2]{\textcolor{blue}{\href{#1}{#2}}}

\newcommand{\Wrapped}[2][c]{%
  \begin{tabular}[#1]{@{}c@{}}#2\end{tabular}}


\title[Другие примеры наборов инструкций]{Архитектура компьютеров\texorpdfstring{\\}{ }Лекция 10. Другие примеры наборов инструкций}

\begin{document}

\begin{frame}
\titlepage
\end{frame}

\begin{frame}
\frametitle{Содержание}
\tableofcontents
\end{frame}

\section{Развитие архитектуры x86}

\begin{frame}{Развитие архитектуры x86}

\pause
\begin{itemize}[<+->]\itemsep=.3cm
      \item 16-разрядные x86-процессоры:\\ Intel 8086 (1978 г.), 8088, 80186, 80286 (1982 г.)

      \item 32-разрядные x86-процессоры:
      \begin{itemize}
        \item Intel 80386 (1985 г.), 80486,
        \item семейство Pentium  (до середины 2000-х),
        \item Intel Core (2006 г);
        \item AMD K5, K6, K7 (Athlon)…
      \end{itemize}

      \item 64-разрядные архитектуры и процессоры:
      \begin{itemize}
          \item Intel Architecture 64 aka IA-64 и процессор Itanium (2001~г.)
          \item AMD64 aka x86-64: AMD Opteron (2003 г.),
          \item Pentium 4 и Xeon с 2004 г., Intel Core 2 (2006 г.), Intel Core i3/i5/i7.
      \end{itemize}
\end{itemize}

\end{frame}

\begin{frame}{Преимущества 64-х разрядов}
\pause
    \begin{itemize}\itemsep=.5cm
        \item Cложнее получить переполнение.

        \item Увеличение производительности программ уже использующих 64-разрядные целые и вещественные числа.

        \item Б\emph{о}льшее адресное пространство.
    \end{itemize}
\end{frame}

\begin{frame}{Регистры в архитектуре x86-64: общая картина}
\begin{columns}
    \column{7cm}\vspace{-.7cm}\screenshotw{7.3cm}{x86-64-registers.png}
    \column{5cm}
\pause\small\begin{itemize}[<+->]
    \item x87~— старая реализация IEEE~754
    \item MMX~— SIMD-инструкции, 64-битные регистры,\\целые операнды:\\
        $1\times64$, $2\times32$, $4\times16$, $8\times8$,
    \item SSE~— SIMD-инструкции, 128-битные регистры, вещественные операнды:\\
        $4\times32$
    \item SSE2~— SIMD-инструкции, 128-битные регистры,
        вещественные или целые операнды:\\
        $2\times64$, $4\times32$, $8\times16$, $16\times8$,
\end{itemize}
\end{columns}
\end{frame}

\begin{frame}{Имена регистров x86 (1): GPE, x87/MMX, special}
\screenshotw{12cm}{x86-registers-1.pdf}
\end{frame}

\begin{frame}{Имена регистров x86 (2): SSE, AVX}
\screenshotw{11.7cm}{x86-registers-2.pdf}
\end{frame}

\subsection{Адресация в архитектуре x86}

\begin{frame}{8086, 8088 (pre-80286)}

\pause
\begin{columns}
  \column{5.5cm}
\begin{itemize}[<+->]
    \item Сегментные регистры:\\
    \m{DS}, \m{CS}, \m{SS} и \m{ES}.

    \item Когда я пишу
    \begin{itemize}
      \item \m{MOV AX, (BX)} — подразумеваю \m{MOV~AX,~(DS:BX)},\\
    %\hphantom{Когда я пишу }
      \item \m{CALL AX} — подразумеваю \m{CALL~CS:AX},

      \item \m{MOV AX, 4(BP)} — подразумеваю \m{MOV~AX,~4(SS:BP)},
    \end{itemize}

    \item 20-битная адресная шина,
        {\small(сегмент\textsubscript{16} <\hspace{0cm}< 4) + смещение\textsubscript{16}}

    \item логический, линейный и физический адреса.
\end{itemize}

  \column{6.5cm}
    \pause
    \vspace{-.3cm}
    \screenshotw{6.3cm}{x86-addressing-real.pdf}
\end{columns}
\end{frame}

\begin{frame}{80286: 24-битная адресная шина}
\pause{}MMU: Memory Management Unit, <<защищённый~режим>>
\screenshotw{10cm}{x86-addressing-protected-2.pdf}
\end{frame}


\begin{frame}{80286: получение по логическому адресу линейного (физического)}

    \pause
    \begin{itemize}[<+->]\itemsep=.5cm
        \item \{селектор дескриптора\}\textsubscript{16} $\leftarrow$
            сегментный регистр,

        \item \{дескриптор сегмента\}\textsubscript{64} $\leftarrow$
            селектор дескриптора,

        \item сегмент\textsubscript{24} $\leftarrow$ дескриптор сегмента,

        \item \{линейный адрес\}\textsubscript{24} $\leftarrow$
            сегмент + смещение\textsubscript{16}.
    \end{itemize}
\end{frame}

\begin{frame}{80386 (x86-32): 32-битная адресная шина, «виртуальная~память»}
 \pause
    \screenshotw{10cm}{i386-address-translation.png}

\end{frame}

\begin{frame}{80386 (x86-32): получение физического адреса}
 \pause
    \screenshotw{10.3cm}{x86-paging-2.pdf}
\end{frame}

\begin{frame}{Преодоление 4Gb-предела}
\begin{itemize}\itemsep=1cm
%\setcounter{enumi}{3}
    \item Pentium Pro (1995 г.): 36-битная адресная шина,\\
    Physical Address Extension (PAE) — многоуровневые таблицы трансляции адресов.
    \pause\item x86-64: расширения PAE, 48-битная адресная шина (256 Tb)
\end{itemize}
\end{frame}

\begin{frame}
\frametitle{Другие нововведения x86-64}
\begin{itemize}\itemsep=.6cm
    \item Два режима работы: Long / Legacy (зависит от ОС);
    \item адресация относительно PC;
    \item бит NX для отдельных страниц виртуальной памяти;
    \item SSE2-инструкции для плавающей точки по умолчанию.
\end{itemize}
\end{frame}

\section{Стековые машины и IJVM}

\begin{frame}{Стековые машины}

    \pause
    %\begin{block}
    \screenshotw{11cm}{stacks.png}
    %\end{block}
    \pause

    %\begin{block}{
    Стек операндов, пример: a1 = a2 + a3:
    %}
    \pause\screenshotw{11cm}{operand-stack.png}
    %\end{block}

    \pause
    \vspace{-.2cm}{\small Стековые машины vs. регистровые машины.}
\end{frame}

\begin{frame}{Модель памяти IJVM}
    \screenshotw{11cm}{ijvm-memory-model.png}
\end{frame}

\begin{frame}{Набор инструкций IJVM}
\screenshotw{8.5cm}{ijvm-isa.pdf}
\end{frame}

%\begin{frame}{INVOKEVIRTIAL}
%\screenshotw{10.5cm}{ijvm-invokev.pdf}
%\end{frame}

\begin{frame}{Компиляция в IJVM}
\screenshotw{11cm}{ijvm-compile.png}
\end{frame}

\section{Инструменты работы на уровне ISA}

\begin{frame}{Флаги компилятора, \m{binutils}}

\begin{itemize}\itemsep=1cm
  \item \m{gcc -S}

  \item \m{objdump}
\end{itemize}

\end{frame}

\end{document}
