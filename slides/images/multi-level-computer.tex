{\small
\tikzset{
>=stealth',
  punktchain/.style={
    rectangle,
    rounded corners,
    % fill=black!10,
    draw=black, very thick,
    text width=15em,
    minimum height=2em,
    text centered,
    top color=white,
    on chain},
  every join/.style={->, thick,shorten >=1pt},
}
\begin{tikzpicture}
  [node distance=.7cm,
  start chain=going below,]
     \node[punktchain, join, bottom color=lime!60] (PL)
         {Языки «высокого уровня»};

     \node[punktchain, join, bottom color=teal!75] (ASM)
         {Язык ассемблера};

     \node[punktchain, join, bottom color=blue!50] (BIN)
         {«Машинный код»\\(уровень набора инструкций, Instruction Set Arch., \textbf{ISA})};

     \node[punktchain, join, bottom color=yellow!60] (MuCode)
         {Микрокод процессора\\(\textbf{микроархитектура})};

     \node[punktchain, join, bottom color=red!60] (DL)
         {Схемы цифровой логики};
\end{tikzpicture}
}
