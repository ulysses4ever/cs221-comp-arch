\newcommand{\h}{%
handout,%
}

\documentclass[
\h%
ucs,ignorenonframetext,hyperref={pdftex,unicode},xcolor=dvipsnames]{beamer}
\usepackage[utf8x]{inputenc}
\usepackage[russian]{babel}
\usepackage[T2A]{fontenc}
\usepackage{tikz}
\usetikzlibrary{shapes,arrows,positioning,chains,calc}
\usepackage{graphicx}
\usepackage{fixltx2e}
\usepackage{paratype}
\usepackage{booktabs}
\usepackage{xcolor}
\usepackage{pbox}

\newcommand{\screenshot}[1]{
\begin{center}
\includegraphics[width=12cm,keepaspectratio]{./images/#1}
\end{center}
}

\newcommand{\screenshotw}[2]{
\begin{center}
\includegraphics[width=#1,keepaspectratio]{./images/#2}
\end{center}
}

\usecolortheme{crane}
\useoutertheme{infolines}
\setbeamertemplate{navigation symbols}{}

\author[А. М. Пеленицын]{А.~М.~Пеленицын\texorpdfstring{\\}{ }
apel@sfedu.ru}

\date{Весна 2016}

\institute[Мехмат ЮФУ]{Южный федеральный университет \texorpdfstring{\\}{ }
Институт математики, механики и компьютерных наук им. И.\,И.~Воровича\texorpdfstring{\\}{ }
Кафедра информатики и вычислительного эксперимента}

\subtitle{}

\AtBeginSection[]
{
\begin{frame}<beamer>
\frametitle{Содержание}
\tableofcontents[currentsection,hideothersubsections]
\end{frame}
}

\AtBeginSubsection[]
{
\begin{frame}<beamer>
\frametitle{Содержание}
\tableofcontents[currentsection,subsectionstyle=show/shaded/hide]
\end{frame}
}

\newcommand{\nspace}{\hspace{0pt}}
\newcommand{\nbdash}{\nobreakdash-\nspace}
\newcommand{\up}{\textsuperscript}
\newcommand{\bslash}{\textbackslash}

\newcommand{\link}[2]{\textcolor{blue}{\href{#1}{#2}}}

\newcommand{\Wrapped}[2][c]{%
  \begin{tabular}[#1]{@{}c@{}}#2\end{tabular}}


\usepackage{tikz}
\usetikzlibrary{arrows,automata}

\usepackage{framed}
\usepackage{tabu}

\newcommand{\p}{\textbf{\textcolor{green}{+}}}
\newcommand{\myminus}{\textbf{\textcolor{red}{–}}}

\title[Цифровая логика: схемы памяти]{Архитектура компьютеров\texorpdfstring{\\}{ }
Лекция 14. Уровень цифровой логики:\texorpdfstring{\\}{ }схемы и шины памяти}

\begin{document}

\begin{frame}
\titlepage
\end{frame}

%\section{Тактовые генераторы (clocks)}
\begin{frame}
\frametitle{Тактовые генераторы (clocks)}
\screenshotw{10cm}{clocks.png}
\end{frame}

\section{Малые последовательностные схемы}

\begin{frame}
\frametitle{SR-защёлка (latch)}
\screenshotw{7cm}{SR-latch.png}
\end{frame}

\begin{frame}
\frametitle{Два устойчивых состояния SR-защёлки}

\vspace{-.2cm}
\screenshotw{12cm}{SR-latch-stable-states.png}

\pause
\centering
\vspace{-.1cm}
\begin{minipage}{8cm}
\begin{framed}
\vspace{-.2cm}
        \parbox{4cm}{Конечный автомат\\
            для SR-защёлки}
        \parbox{3cm}{\newcommand{\myscale}{0.7}

\begin{tikzpicture}[->,>=stealth',shorten >=1pt,auto,node distance=2.8cm,%
                    semithick,scale=\myscale, every node/.style={scale=\myscale}]
  %\tikzstyle{every state}=[fill=red,draw=none,text=white]

  \node[state] (q0)                    {$Q=0$};
  \node[state] (q1) [right of=q0] {$Q=1$};
  \path (q0.north east) edge [bend left]   node {$S$}  (q1.north west)
         (q0)            edge [loop above]  node {$R$}   (q0)
         (q1)            edge [loop above]  node {$S$}   (q1)
         (q1.south west) edge [bend left]   node {$R$}  (q0.south east)
%            edge [bend left]  node {1,0,R} (E)
  ;
\end{tikzpicture}
}
\vspace{-.2cm}
\end{framed}
\end{minipage}
\end{frame}

\begin{frame}
\frametitle{Синхронная SR-защёлка}
\screenshotw{12cm}{SR-synch.png}
\end{frame}

\begin{frame}
\frametitle{Синхронная D-защёлка}
\screenshotw{12cm}{D-synch-latch.png}
\end{frame}

%\begin{frame}
%\frametitle{Синхронная JK-защёлка}
%\screenshotw{11cm}{JK-latch.png}
%\end{frame}

\begin{frame}
\frametitle{Триггеры (flip-flops)}\vspace{-.7cm}
\begin{columns}
    \column{5cm} \pause\screenshotw{5cm}{edge-generator.png}
    \column{7cm} \pause\screenshotw{7cm}{edge-generator-timing.png}
\end{columns}
\end{frame}

\section{Средние последовательностные схемы}

\begin{frame}
\frametitle{Схема 4-битного регистра}
\screenshotw{5.5cm}{register-curcuit.jpg}
\end{frame}

\begin{frame}[plain]
\frametitle{4x3-схема памяти на D-триггерах}
\screenshotw{7.1cm}{4x3-memory-chip.pdf}
\end{frame}

\section{VLSI-последовательностные схемы}

\begin{frame}
\frametitle{4 Mbit-схемы памяти}
\screenshotw{11cm}{large-memory-chips-1.png}
\end{frame}

\begin{frame}
\frametitle{512 Mbit-схемы памяти}
\screenshotw{11cm}{large-memory-chips-2.png}
\end{frame}

\section{Разнообразие технологий памяти}

\begin{frame}
\frametitle{Типы микросхем ОЗУ}
\begin{itemize}[<+->]\itemsep=.5cm
    \item Статическая и динамическая память (SRAM vs. DRAM).
    \item Синхронная динамическая память (SDRAM).
    \item Double Data Rate (DDR) SDRAM.
\end{itemize}
\end{frame}

\begin{frame}
\frametitle{Сравнительная таблица: типы микросхем ОЗУ и ПЗУ}
%\screenshotw{12cm}{memory-chips.png}
\scriptsize
\begin{table}
\renewcommand{\arraystretch}{1.5}
\begin{tabu} to \textwidth {lllX[.4]X[.4]X}
\toprule
\textbf{№} & \multicolumn{1}{c}{\textbf{Тип}} & \multicolumn{1}{c}{\textbf{Операции}} & \centering\textbf{Побайтовая}\newline \textbf{запись} & \centering\textbf{Энерго-}\newline\textbf{НЕзависимость} & \multicolumn{1}{c}{\textbf{Области применения}}                                      \\
\midrule
1 & SRAM                             & R/W                                   & \p                                                                                        & \myminus                                                                                           & Регистры, кэш                                                                        \\
2 & DRAM                             & R/W                                   & \p                                                                                        & \myminus                                                                                           & Оперативная память (раньше)                \\
3 & SDRAM                            & R/W                                   & \p                                                                                        & \myminus                                                                                           & Оперативная память (сегодня)               \\
4 & ROM                              & RO                                    & \myminus                                                                                       & \p                                                                                            & Крупные партии \newline микроконтроллеров и т. п.  \\
5 & PROM                             & RO                                    & \myminus                                                                                       & \p                                                                                            & Мелкие партии \newline микроконтроллеров и т. п.   \\
6 & EPROM                            & \textit{Больше} R                     & \myminus                                                                                       & \p                                                                                            & Прототипирование \newline микроконтроллеров и т. п. \\
7 & EEPROM                           & Больше R                              & \p                                                                                        & \p                                                                                            & Прототипирование\newline микроконтроллеров и т. п. \\
8 & Флеш                             & R/W                                   & \myminus                                                                                       & \p                                                                                            & Камеры, смартфоны…\\
\bottomrule
\end{tabu}
\end{table}
\end{frame}

\section{Память DDR}
\begin{frame}{Организация модуля памяти DDR, терминология}
\pause
\begin{itemize}%\itemsep=.3cm
    \item Модуль — печатная плата с $N$ чипами.

    \pause\item Матричный доступ внутри одного чипа: {\color{red} \m{row} x \m{col}}.

    \pause\item Третье измерение внутри одного чипа:\\ «{\color{red}банк}»  — помнит «открытую» (выбранную) строку.

    \pause\item «Ранг» (rank) — группа чипов на планке, которые предоставляют доступ к 64-разрядному значению.


    \pause\item Пример: габариты 1-Гб планки
    \begin{itemize}
        \item чип: 64M×4,
        \item всего 32 чипа,
        \item два ранга.
    \end{itemize}


    \pause\item Основные команды для $k$-го банка памяти:
    \begin{itemize}
        \item ACTIVATE — «открывает» $i$-ую строку;
        \item READ / WRITE — читает / пишет в $j$-й столбец;
        \item PRECHARGE — «закрывает» текущую открытую строку.
    \end{itemize}
\end{itemize}
\end{frame}

\begin{frame}
\frametitle{Пример: 12 циклов работы на шине памяти DDR3}
ACTIVATE = 2 такта, \hfill READ = 1 такт, \hfill PRECHARGE = 2 такта.
\screenshotw{12cm}{pipelined-memory.png}
\end{frame}


\end{document}
