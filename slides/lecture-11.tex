\newcommand{\h}{%
handout,%
}

\documentclass[
\h%
ucs,ignorenonframetext,hyperref={pdftex,unicode},xcolor=dvipsnames]{beamer}
\usepackage[utf8x]{inputenc}
\usepackage[russian]{babel}
\usepackage[T2A]{fontenc}
\usepackage{tikz}
\usetikzlibrary{shapes,arrows,positioning,chains,calc}
\usepackage{graphicx}
\usepackage{fixltx2e}
\usepackage{paratype}
\usepackage{booktabs}
\usepackage{xcolor}
\usepackage{pbox}

\newcommand{\screenshot}[1]{
\begin{center}
\includegraphics[width=12cm,keepaspectratio]{./images/#1}
\end{center}
}

\newcommand{\screenshotw}[2]{
\begin{center}
\includegraphics[width=#1,keepaspectratio]{./images/#2}
\end{center}
}

\usecolortheme{crane}
\useoutertheme{infolines}
\setbeamertemplate{navigation symbols}{}

\author[А. М. Пеленицын]{А.~М.~Пеленицын\texorpdfstring{\\}{ }
apel@sfedu.ru}

\date{Весна 2016}

\institute[Мехмат ЮФУ]{Южный федеральный университет \texorpdfstring{\\}{ }
Институт математики, механики и компьютерных наук им. И.\,И.~Воровича\texorpdfstring{\\}{ }
Кафедра информатики и вычислительного эксперимента}

\subtitle{}

\AtBeginSection[]
{
\begin{frame}<beamer>
\frametitle{Содержание}
\tableofcontents[currentsection,hideothersubsections]
\end{frame}
}

\AtBeginSubsection[]
{
\begin{frame}<beamer>
\frametitle{Содержание}
\tableofcontents[currentsection,subsectionstyle=show/shaded/hide]
\end{frame}
}

\newcommand{\nspace}{\hspace{0pt}}
\newcommand{\nbdash}{\nobreakdash-\nspace}
\newcommand{\up}{\textsuperscript}
\newcommand{\bslash}{\textbackslash}

\newcommand{\link}[2]{\textcolor{blue}{\href{#1}{#2}}}

\newcommand{\Wrapped}[2][c]{%
  \begin{tabular}[#1]{@{}c@{}}#2\end{tabular}}


\title[Уровень микроархитектуры и Mic-1]{Архитектура компьютеров\texorpdfstring{\\}{ }Лекция 11. Уровень микроархитектуры:
Mic-1~и~микропрограммное управление}

\begin{document}

\begin{frame}
\titlepage
\end{frame}

\section{Введение}

\begin{frame}{Современные многоуровневые машины}
\center
{\small
\tikzset{
>=stealth',
  punktchain/.style={
    rectangle,
    rounded corners,
    % fill=black!10,
    draw=black, very thick,
    text width=15em,
    minimum height=2em,
    text centered,
    top color=white,
    on chain},
  every join/.style={->, thick,shorten >=1pt},
}
\begin{tikzpicture}
  [node distance=.7cm,
  start chain=going below,]
     \node[punktchain, join, bottom color=lime!60] (PL)
         {Языки «высокого уровня»};

     \node[punktchain, join, bottom color=teal!75] (ASM)
         {Язык ассемблера};

     \node[punktchain, join, bottom color=blue!50] (BIN)
         {«Машинный код»\\(уровень набора инструкций, Instruction Set Arch., \textbf{ISA})};

     \node[punktchain, join, bottom color=yellow!60] (MuCode)
         {Микрокод процессора\\(\textbf{микроархитектура})};

     \node[punktchain, join, bottom color=red!60] (DL)
         {Схемы цифровой логики};
\end{tikzpicture}
}

\end{frame}

\begin{frame}{Набор инструкций IJVM}
\screenshotw{8.5cm}{ijvm-isa.pdf}
\end{frame}

\begin{frame}{Компиляция в IJVM}
\screenshotw{12cm}{ijvm-compile.png}
\end{frame}

\begin{frame}{Организация центрального процессора}
  	\screenshotw{7cm}{cpu.pdf}
\end{frame}

\begin{frame}{Немного об АЛУ}
  \screenshotw{9cm}{alu.pdf}

\pause
\begin{itemize}
  \item Основные \emph{бинарные} функции: сложение, вычитание, AND, OR…
  \item Основные \emph{унарные} функции: взятие противоположного,
  \hphantom{Основные \emph{унарные} функции: }отрицание, инкремент, декремент… %
\end{itemize}

\end{frame}

\begin{frame}
\frametitle{Содержание}
\tableofcontents
\end{frame}

\section{Тракт данных Mic-1}

\begin{frame}[plain]
\frametitle{Тракт данных Mic-1}
    \vspace{-.2cm}\screenshotw{5.7cm}{datapath.pdf}
\end{frame}

%~ \begin{frame}
%~ \frametitle{Синхронизация тракта данных Mic-1}
%~ \begin{columns}
        %~ \column{4cm}
    %~ \screenshotw{3.5cm}{datapath-light.pdf}

        %~ \column{9cm}
    %~ \screenshotw{8.7cm}{datapath-timing-crpd-c1.png}
%~ \end{columns}
%~ \end{frame}

\begin{frame}{Связь с оперативной памятью}

\footnotesize
\begin{columns}
        \column{4cm}
    \onslide<1->{\screenshotw{3.5cm}{datapath-light.pdf}}

        \column{9cm}
\onslide<2->{\begin{itemize}
    \item Регистры для обработки данных:
        \begin{itemize}\footnotesize
            \item MDR (Memory Data Register) — 32-bit wide,
            \item MAR (Memory Address Register) — word-oriented, 32-bit wide.
        \end{itemize}
\end{itemize}}

\onslide<3->{\screenshotw{8cm}{MAR-to-bus.png}}

\onslide<4->{\begin{itemize}
    \item Регистры для обработки инструкций:
        \begin{itemize}\footnotesize
            \item PC (Program Counter) — byte-oriented, 32-bit wide,
            \item MBR (Memory Buffer Register) — 8-bit wide.
        \end{itemize}}

\onslide<5->{
\item Управляющие сигналы: rd / wr / fetch.
\end{itemize}}
\end{columns}

\end{frame}

\section{Микропрограммное управление Mic-1}

\begin{frame}
\frametitle{Управляющие (командные) сигналы тракта данных}
\begin{columns}
        \column{3.5cm}
    %\vspace{-.9cm}
    \screenshotw{3.5cm}{datapath-light.pdf}

        \column{9cm}

\pause\begin{itemize}\small
    \item 9 сигналов для загрузки одного из регистров на шину B;
    \item 9 сигналов для записи в некоторые регистры значения с шины C;
    \item 8 сигналов для управления АЛУ и схемой сдвига;
    \item 2 сигнала: запись или чтение слова из памяти;
    \item 1 сигнал: необходимость загрузки (fetch) очередной команды.
\end{itemize}

\pause\begin{itemize}
\item Всего управляющих сигналов: \pause 29.
\pause \item Можно ли сэкономить?
\pause \item Идея: закодировать номер сигнала загрузки на шину B \pause ⇒ 24 сигнала.
\end{itemize}
\end{columns}

\end{frame}

\begin{frame}
\frametitle{Формат микрокоманды}
\small 24 бита команды + 12 бит переход к следующей = 36 бит — в ПЗУ на ЦП.

\pause

\screenshotw{11cm}{microcommand.png}

\pause

\vspace{-2.5cm}
Поля \texttt{JAM} и \texttt{NEXT\_ADDRESS} реализуют логику \\
    \alert{управления последовательностью микрокоманд}\\
    (\alert{sequencer}).

\pause

\vspace{.2cm}
Основные способы определения адреса следующей:
\begin{enumerate}\itemsep=0cm
  \item Целиком зашито в поле \texttt{NEXT\_ADDRESS}.
  \item Целиком берётся из \texttt{MBR}.
  \item Берётся из \texttt{NEXT\_ADDRESS}, но учитывает флаги АЛУ.
\end{enumerate}
\end{frame}

\begin{frame}
\frametitle{Определение адреса следующей микрокоманды (\m{MPC})}

    Пример
    \screenshotw{12cm}{jamz.png}

    \pause
    В общем случае:

    \pause
    {\small \ttfamily
    MPC[0..7] = JMPC ? MBR : NEXT\_ADDRESS[0..7]}

    \pause
    {\small \ttfamily MPC[8] = ( JAMZ \& Z ) | ( JAMN \& N )
        | \color{gray}{NEXT\_ADDRESS[8]}}
\end{frame}

\begin{frame}[plain]
\frametitle{Микроархитектура Mic-1}
\vspace{-.3cm}
\screenshotw{8.5cm}{Mic-1.png}
\end{frame}

\section{Микроассемблер для Mic-1 и реализация IJVM}

\begin{frame}{Микроассемблер MAC}
    \alert{Как удобно записывать микрокоманды?}\\
    {\small ReadReg = SP, ALU = INC, WSP, Read, NEXT\_ADDRESS = 122? \pause — Nope!}
    \pause

\begin{columns}
        \column{5cm}
    \begin{block}{Арифметика и память}
    \begin{itemize}
        \item \pause SP = SP + 1; rd
        \item \pause MDR = SP
        \item \pause MDR = H + SP
        \item \pause SP = MDR = SP + 1
        \item \pause MDR = SP + MDR \pause — нельзя!
        \item \pause некорректная пара:\\
            MAR = SP; rd \\
            MDR = H
    \end{itemize}
    \end{block}

        \column{6.5cm}
    %\vspace{.5cm}
    \pause \begin{block}{Переходы}
    \begin{itemize}
        \item \pause goto \textit{label}
        \item \pause учёт флагов Z/N:\\
            {\small Z = TOS ; if (Z) goto L1; else goto L2}
        \item \pause goto (MBR)
    \end{itemize}
    \end{block}

\end{columns}
\end{frame}


\begin{frame}{Примеры микропрограммных реализаций инструкций}
    \ttfamily\small

\pause
«Основной цикл»:\\
\m{Main1: PC = PC + 1; fetch; goto (MBR)}

\begin{columns}
        \column{5.5cm}

    \pause
    \begin{block}{\texttt{IADD (0x60)}}
    \pause
    \begin{enumerate}
        \item MAR = SP = SP — 1; rd
        \item H = TOS
        \item MDR = TOS = H + MDR; wr; goto Main1
    \end{enumerate}
    \end{block}

        \column{5.5cm}
    \pause
    \begin{block}{\texttt{ILOAD k}}
    \begin{enumerate}
        \item H = LV
        \item MAR = MBRU + H; rd
        \item MAR = SP = SP + 1
        \item PC = PC + 1; fetch; wr
        \item TOS = MDR; goto Main1
    \end{enumerate}
    \end{block}

\end{columns}
\end{frame}

\end{document}

