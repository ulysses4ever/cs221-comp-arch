\newcommand{\h}{handout,%
}

\documentclass[
\h%
ucs,ignorenonframetext,hyperref={pdftex,unicode},xcolor=dvipsnames]{beamer}
\usepackage[utf8x]{inputenc}
\usepackage[russian]{babel}
\usepackage[T2A]{fontenc}
\usepackage{tikz}
\usetikzlibrary{shapes,arrows,positioning,chains,calc}
\usepackage{graphicx}
\usepackage{fixltx2e}
\usepackage{paratype}
\usepackage{booktabs}
\usepackage{xcolor}
\usepackage{pbox}

\newcommand{\screenshot}[1]{
\begin{center}
\includegraphics[width=12cm,keepaspectratio]{./images/#1}
\end{center}
}

\newcommand{\screenshotw}[2]{
\begin{center}
\includegraphics[width=#1,keepaspectratio]{./images/#2}
\end{center}
}

\usecolortheme{crane}
\useoutertheme{infolines}
\setbeamertemplate{navigation symbols}{}

\author[А. М. Пеленицын]{А.~М.~Пеленицын\texorpdfstring{\\}{ }
apel@sfedu.ru}

\date{Весна 2016}

\institute[Мехмат ЮФУ]{Южный федеральный университет \texorpdfstring{\\}{ }
Институт математики, механики и компьютерных наук им. И.\,И.~Воровича\texorpdfstring{\\}{ }
Кафедра информатики и вычислительного эксперимента}

\subtitle{}

\AtBeginSection[]
{
\begin{frame}<beamer>
\frametitle{Содержание}
\tableofcontents[currentsection,hideothersubsections]
\end{frame}
}

\AtBeginSubsection[]
{
\begin{frame}<beamer>
\frametitle{Содержание}
\tableofcontents[currentsection,subsectionstyle=show/shaded/hide]
\end{frame}
}

\newcommand{\nspace}{\hspace{0pt}}
\newcommand{\nbdash}{\nobreakdash-\nspace}
\newcommand{\up}{\textsuperscript}
\newcommand{\bslash}{\textbackslash}

\newcommand{\link}[2]{\textcolor{blue}{\href{#1}{#2}}}

\newcommand{\Wrapped}[2][c]{%
  \begin{tabular}[#1]{@{}c@{}}#2\end{tabular}}


\title[Оперативная память]{Архитектура компьютеров\texorpdfstring{\\}{ }Лекция 5. Оперативная память}

\begin{document}

\begin{frame}
\titlepage
\end{frame}

\begin{frame}
\frametitle{Содержание}
\tableofcontents[hideallsubsections]
\end{frame}

\section {Организация оперативной памяти}

\begin{frame}
\frametitle{Биты}
\pause{\begin{itemize}[<+->]
    \item Почему двоичная система?
    \item Двоично-десятичный код (binary-coded decimals, BCD).
    \item Воображаемый пример: десяти-разрядные запоминающие устройства.
    \item Квантовые биты: двумерное комплексное линейное пространство состояний вместо двух состояний.
\end{itemize}}
\end{frame}

\begin{frame}
\frametitle{Понятие «ячейки памяти»}
\vspace{-.2cm}\screenshotw{10cm}{bytes-example.png}
\pause{\vspace{-.4cm}\begin{itemize}
    \item Адресное пространство.
\end{itemize}}
\end{frame}

\begin{frame}[fragile]
\frametitle{Примеры}
%\screenshotw{8cm}{byte-variants.png}
\centering
\begin{tabular}{lp{2.1cm}}
    \toprule
    \textbf{Компьютер} & \textbf{Число битов в ячейке}\\
    \midrule
    Burroughs B1700  & 1 \\
    IBM PC           & 8 \\
    DEC PDP-8        & 12\\
    IBM 1130         & 16\\
    DEC PDP-15       & 18\\
    XDS 940          & 24\\
    Electrologica X8 & 27\\
    XDS Sigma 9      & 32\\
    Honeywell 6180   & 36\\
    CDC 3600         & 48\\
    CDC Cyber        & 60\\
    \bottomrule
\end{tabular}

\pause{\begin{itemize}
    \item Cell — location — byte — octet — char.
\end{itemize}}
\end{frame}

\section {Порядок байт}

\begin{frame}
\frametitle{Порядок байт в слове (Endianness)}

\begin{itemize}[<+->]
    \item Память байт-ориентирована, процессор слово-ориентирован.
    \item Как представлено число 1 в оперативной памяти\\
           32-разрядной машины?
\end{itemize}

\pause
%\screenshotw{11cm}{endianness.png}

{\centering\footnotesize
\begin{tabular}{ccccc}
    \toprule
    \textbf{Порядок}
        & \textbf{\Wrapped{Первый байт\\(младший адрес)}}
        & \textbf{\Wrapped{Средние\\байты}}
        & \textbf{\Wrapped{Последний байт\\(старший адрес)}}
        & \textbf{\Wrapped{Число 1000 (0x3E8)\\в двух байтах}}
        \\
    \midrule
    \Wrapped{Прямой\\ \scriptsize(big-endian)}
        & старший
        & …
        & младший
        & 03 E8
        \\
    \Wrapped{Обратный\\ \scriptsize(little-endian)}
        & младший
        & …
        & старший
        & E8 03
        \\
    \bottomrule
\end{tabular}}

\pause\small
\begin{block}{История}
\begin{itemize}
    \item Danny Cohen (1980-04-01): “On Holy Wars and a Plea for Peace”
    \item NUXI problem
\end{itemize}
\end{block}

\end{frame}


\begin{frame}{Пример: 0A0B0C0D$_{\text{h}}$}
\begin{columns}
    \column{10.5cm}
\begin{itemize}\itemsep=10pt
    \item 8-битный байт, прямой (big) порядок:\hfill |0A|0B|0C|0D|;
    \item 8-битный байт, обратный (little) порядок:\hfill |0D|0C|0B|0A|;
    \item 16-битный байт, прямой (big) порядок:\hfill |0A0B|0C0D|;
    \item 16-битный байт, обратный (little) порядок:\hfill |0C0D|0A0B|.
\end{itemize}
    \column{0cm}
\end{columns}
\end{frame}

\begin{frame}{Примеры реализации}
\begin{itemize}[<+->]
    \item Прямой порядок (aka \emph{big-endian order} aka \emph{network order}):\\
    IBM System 360/370/.../z, Motorolla 68000, Atmel AVR32;

    \item обратный порядок (aka \emph{little-endian order}):\\
    x86(-64), Atmel AVR, DEC VAX;

    \item переключаемый порядок (bi-endian): ARMv3+, SPARC V9+, MIPS;

    \item middle-endian: PDP-11.
\end{itemize}
\end{frame}

\begin{frame}{Какой порядок выбрать?}
\pause
\begin{itemize}\itemsep=10pt
    \item Прямой порядок привычней для человека;
    \item обратный порядок упрощает адресацию;
    \item обратный порядок упрощает реализацию арифметики\\
         в простых машинах.
\end{itemize}
\end{frame}

\section {Сверхоперативная память: кэш-память}

\begin{frame}{Устройство и основные идеи}
\screenshotw{7cm}{cache-memory.png}
\pause{\begin{itemize}[<+->]
    \item \link%
    {http://en.wikipedia.org/wiki/Von_Neumann_architecture\#Von_Neumann_bottleneck}%
    {Wikipedia: Von Neumann bottleneck};
    \item принцип локальности;
    \item кэш-линейки (cache lines);
    \item формула: если $c$~— среднее время доступа к кэш-памяти,
           \\$m$~— к основной памяти, а $h$~— доля кэш-промахов, то:

    \begin{center}
    среднее время доступа к данным = \onslide<6->{$c + h \cdot m$}
    \end{center}

    \item<7-> split cache.
\end{itemize}}
\end{frame}

\begin{frame}{Вопросы проектирования кэш-памяти}
\begin{enumerate}
    \item Размер кэш-памяти.
    \item Размер кэш-линеек.
    \item Использование сплит-кэша.
    \item Организация: алгоритм сброса и адресация.
    \item Уровни кэш-памяти.
\end{enumerate}
\end{frame}

\end{document}

%%%%%%%%%%%%%%%%%%%%%%%%%%%%%%  КОНЕЦ  %%%%%%%%%%%%%%%%%%%%%%%%%%%%%%%

\begin{frame}{Пути преодоления Von Neumann bottleneck}
\begin{columns}
    \column{7cm}
\screenshotw{7cm}{pipeline-4.pdf}
    \column{6cm}
\begin{enumerate}
    \item Кэш-память.
    \item Сплит-кэш.
    \item Прогнозирование ветвлений\\
        (\link{http://en.wikipedia.org/wiki/Branch_predictor}%
        {branch prediction}).
    \item Память-черновик\\(scratchpad memory).

\end{enumerate}
\end{columns}
\end{frame}
