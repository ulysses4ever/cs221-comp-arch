\newcommand{\h}{%
handout,%
}

\documentclass[
\h%
ucs,ignorenonframetext,hyperref={pdftex,unicode},xcolor=dvipsnames]{beamer}
\usepackage[utf8x]{inputenc}
\usepackage[russian]{babel}
\usepackage[T2A]{fontenc}
\usepackage{tikz}
\usetikzlibrary{shapes,arrows,positioning,chains,calc}
\usepackage{graphicx}
\usepackage{fixltx2e}
\usepackage{paratype}
\usepackage{booktabs}
\usepackage{xcolor}
\usepackage{pbox}

\newcommand{\screenshot}[1]{
\begin{center}
\includegraphics[width=12cm,keepaspectratio]{./images/#1}
\end{center}
}

\newcommand{\screenshotw}[2]{
\begin{center}
\includegraphics[width=#1,keepaspectratio]{./images/#2}
\end{center}
}

\usecolortheme{crane}
\useoutertheme{infolines}
\setbeamertemplate{navigation symbols}{}

\author[А. М. Пеленицын]{А.~М.~Пеленицын\texorpdfstring{\\}{ }
apel@sfedu.ru}

\date{Весна 2016}

\institute[Мехмат ЮФУ]{Южный федеральный университет \texorpdfstring{\\}{ }
Институт математики, механики и компьютерных наук им. И.\,И.~Воровича\texorpdfstring{\\}{ }
Кафедра информатики и вычислительного эксперимента}

\subtitle{}

\AtBeginSection[]
{
\begin{frame}<beamer>
\frametitle{Содержание}
\tableofcontents[currentsection,hideothersubsections]
\end{frame}
}

\AtBeginSubsection[]
{
\begin{frame}<beamer>
\frametitle{Содержание}
\tableofcontents[currentsection,subsectionstyle=show/shaded/hide]
\end{frame}
}

\newcommand{\nspace}{\hspace{0pt}}
\newcommand{\nbdash}{\nobreakdash-\nspace}
\newcommand{\up}{\textsuperscript}
\newcommand{\bslash}{\textbackslash}

\newcommand{\link}[2]{\textcolor{blue}{\href{#1}{#2}}}

\newcommand{\Wrapped}[2][c]{%
  \begin{tabular}[#1]{@{}c@{}}#2\end{tabular}}


%\AtBeginSection[]{}

\title[Уровень микроархитектуры (2)]{Архитектура компьютеров\texorpdfstring{\\}{ }Лекция 12. Уровень микроархитектуры:\texorpdfstring{\\}{ }
дизайн и оптимизация}

\begin{document}

\begin{frame}
\titlepage
\end{frame}

%\begin{frame}
%\frametitle{Содержание}
%\tableofcontents[hideallsubsections]
%\end{frame}

%\section{Разработка уровня микроархитектуры}

\begin{frame}[plain,fragile]{Напоминание: Mic-1}
    \ttfamily\small


\begin{columns}
        \column{4cm}
\vspace{-.5cm}
\screenshotw{4cm}{datapath-light.pdf}

        \column{7.5cm}
\pause
«Основной цикл»:\\
{\footnotesize \m{Main1: PC = PC + 1; fetch; goto (MBR)}}
    \pause
    \begin{block}{\texttt{ILOAD k}}
    \pause
    \begin{enumerate}[<+->]
        \item H = LV
        \item MAR = MBRU + H; rd
        \item MAR = SP = SP + 1
        \item PC = PC + 1; fetch; wr
        \item TOS = MDR; goto Main1
    \end{enumerate}
    \end{block}

\end{columns}
\end{frame}

\begin{frame}{Ускорение работы Mic-1}
\begin{columns}
    \column{6cm}
\begin{block}{Основные параметры}
\pause
\begin{enumerate}[<+->]
    \item Число циклов на
        одну инструкцию.
    \item Длина цикла.
    \item Параллельность исполнения.
\end{enumerate}
\end{block}
%:

\pause
    \column{5.5cm}
\begin{block}{Реализация ускорения}
\begin{itemize}%[<+->]
    \item Блок предварительной выборки инструкций (IFU),
    \item 3-шинная архитектура,
    \item конвейеризация.
\end{itemize}
\end{block}
%:
\end{columns}

\pause
\begin{block}{Другие возможности ускорения}
\begin{itemize}[<+->]
    \item Спекулятивное исполнение и прогнозирование ветвлений,
    \item переименование регистров,
    \item изменение последовательности инструкций,
    \item кэш-память.
\end{itemize}
\end{block}
\end{frame}

%\begin{frame}[plain]
%\frametitle{Реализация некоторых идей ускорения: Mic-3}
%\vspace{-.2cm}
%\screenshotw{7.2cm}{mic-3.png}
%\end{frame}

\section{Сокращение длины пути исполнения}

\begin{frame}{Слияние заголовка цикла и микропрограммы}
\small
\pause
\begin{block}{Стандартный путь исполнения микропрограммы}
\pause
\texttt{Main1: PC = PC + 1; fetch; goto (MBR)}\\
\ldots\\
\ldots\texttt{; goto Main1}
\end{block}
\pause
\begin{columns}
    \column{5.5cm}
    \begin{block}{Прямолинейная реализация \texttt{POP}}
    \pause
    \begin{enumerate}
        \item \texttt{MAR = SP = SP - 1; rd}
        \item \texttt{}
        \item \texttt{TOS = MDR; goto Main1}
    \end{enumerate}
    \end{block}

    \pause
    \column{6cm}
    \begin{block}{Реализация \texttt{POP} через слияние}
    \begin{enumerate}
        \item \texttt{MAR = SP = SP - 1; rd}
        \item \texttt{PC = PC + 1; fetch}
        \item \texttt{TOS = MDR; goto (MBR)}
    \end{enumerate}
    \end{block}
\end{columns}
\end{frame}

\begin{frame}{3-шинная микроархитектура}
\small\pause
\begin{columns}
    \column{5.5cm}
    \begin{block}{Реализация \texttt{ILOAD} на Mic-1}
    \pause
    \begin{enumerate}%[<+->]
        \item \texttt{H = LV}
        \item \texttt{MAR = MBRU + H; rd}
        \item \texttt{MAR = SP = SP + 1}
        \item \texttt{PC = PC + 1; fetch; wr}
        \item \texttt{TOS = MDR; goto Main1}
    \end{enumerate}
    \end{block}

    \pause
    \column{6cm}
    \begin{block}{Реализация \texttt{ILOAD} на Mic-2}
    \begin{enumerate}
        \item \texttt{MAR = MBRU + LV; rd}
        \item \texttt{MAR = SP = SP + 1}
        \item \texttt{PC = PC + 1; fetch; wr}
        \item \texttt{TOS = MDR; goto~Main1}
    \end{enumerate}
    \end{block}
\end{columns}
\end{frame}

\section{Блок предварительной выборки инструкций (IFU)}

\begin{frame}{Логическая схема блока IFU}
\screenshotw{11cm}{IFU.pdf}
\end{frame}

\begin{frame}{Конечный автомат для регистра сдвига в IFU}
\screenshotw{12cm}{IFU-FSM.pdf}
\end{frame}

\begin{frame}{Сравнение реализации \texttt{ILOAD k}}
\small
\begin{columns}
    \column{5.5cm}
    \begin{block}{Старая реализация \texttt{ILOAD k}}
    \begin{enumerate}
        \item \texttt{MAR = MBRU + LV; rd}
        \item \texttt{MAR = SP = SP + 1}
        \item \texttt{PC = PC + 1; fetch; wr}
        \item \texttt{TOS = MDR; goto~Main1}
    \end{enumerate}
    \end{block}

    \pause
    \column{6cm}
    \begin{block}{Новая реализация \texttt{ILOAD k}}
    \begin{enumerate}
        \item \texttt{MAR = LV + MBR1U; rd}
        \item \texttt{MAR = SP = SP + 1}
        \item \texttt{TOS = MDR; wr; goto~(MBR1)}
    \end{enumerate}
    \end{block}
\end{columns}

\vspace{1cm}
Вывод: путь сократился на 2 цикла.
\end{frame}

\begin{frame}{Сравнение реализации \texttt{IINC k n}}
\begin{columns}
    \column{5.7cm}
    \begin{block}{Старая реализация \texttt{IINC k n}}
    \pause
    \begin{enumerate}
        \item \texttt{H = LV}
        \item \texttt{MAR = MBRU + H; rd}
        \item \texttt{PC = PC + 1; fetch}
        \item \texttt{H = MDR}
        \item \texttt{PC = PC + 1; fetch}
        \item \texttt{MDR = MBR + H; wr; goto Main1}
    \end{enumerate}
    \end{block}

    \pause
    \column{5.7cm}
    \begin{block}{Новая реализация \texttt{IINC k n}}
    \begin{enumerate}
        \item \texttt{MAR = LV + MBR1U; rd}
        \item \texttt{H = MBR1}
        \item \texttt{MDR = MDR + H; wr; goto (MBR1)}
    \end{enumerate}
    \end{block}
\end{columns}
\end{frame}

\section{Конвейер}

\begin{frame}[plain]{Микроархитектура Mic-3}
\vspace{-.2cm}
\screenshotw{7.5cm}{Mic-3.pdf}
\end{frame}

\begin{frame}[plain]{Принцип работы конвейера Mic-3}
\vspace{-.2cm}
\screenshotw{8.3cm}{Mic-3-pipeline.pdf}
\end{frame}

\begin{frame}{Пример: инструкция \texttt{SWAP}}
\pause
\begin{columns}
    \column{6cm}
\begin{block}{Старая реализация (Mic-1)}
\pause
    \begin{enumerate}
        \item \texttt{MAR = SP - 1; rd}
        \item \texttt{MAR = SP}
        \item \texttt{H = MDR; wr}
        \item \texttt{MDR = TOS}
        \item \texttt{MAR = SP - 1; wr}
        \item \texttt{TOS = H; goto Main1}
    \end{enumerate}
\end{block}
    \column{0cm}
\end{columns}
\end{frame}

\begin{frame}{Конвейерная реализация \texttt{SWAP}}
\screenshotw{12cm}{Swap-pipelined.pdf}
\end{frame}

\section{Немного о практике}

\begin{frame}{Задача и схема работы}
\screenshotw{11.5cm}{mic-1-lab.pdf}
\end{frame}

\begin{frame}{Симулятор Mic-1}
\screenshotw{9cm}{mic-1-lab-sim.png}
\end{frame}

\end{document}
