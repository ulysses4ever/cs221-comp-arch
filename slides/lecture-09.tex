\newcommand{\h}{%
handout,%
}

\documentclass[
\h%
ucs,ignorenonframetext,hyperref={pdftex,unicode},xcolor=dvipsnames]{beamer}
\usepackage[utf8x]{inputenc}
\usepackage[russian]{babel}
\usepackage[T2A]{fontenc}
\usepackage{tikz}
\usetikzlibrary{calc,trees,positioning,arrows,chains,shapes.geometric,%
    decorations.pathreplacing,decorations.pathmorphing,shapes,%
    matrix,shapes.symbols}
\usepackage{graphicx}
\usepackage{fixltx2e}
\usepackage{paratype}
\usepackage{booktabs}
\usepackage{xcolor}
\usepackage{pbox}

\usepackage[font=itshape]{quoting}

\newcommand{\screenshot}[1]{
\begin{center}
\includegraphics[width=12cm,keepaspectratio]{./images/#1}
\end{center}
}

\newcommand{\screenshotw}[2]{
\begin{center}
\includegraphics[width=#1,keepaspectratio]{./images/#2}
\end{center}
}

\usecolortheme{crane}
\useoutertheme{infolines}
\setbeamertemplate{navigation symbols}{}

\author[А. М. Пеленицын]{А.~М.~Пеленицын\texorpdfstring{\\}{ }
apel@sfedu.ru}

\date{Весна 2016}

\institute[Мехмат ЮФУ]{Южный федеральный университет \texorpdfstring{\\}{ }
Институт математики, механики и компьютерных наук им. И.\,И.~Воровича\texorpdfstring{\\}{ }
Кафедра информатики и вычислительного эксперимента}

\subtitle{}

\AtBeginSection[]
{
\begin{frame}<beamer>
\frametitle{Содержание}
\tableofcontents[currentsection,hideothersubsections]
\end{frame}
}

\AtBeginSubsection[]
{
\begin{frame}<beamer>
\frametitle{Содержание}
\tableofcontents[currentsection,subsectionstyle=show/shaded/hide]
\end{frame}
}

\newcommand{\nspace}{\hspace{0pt}}
\newcommand{\nbdash}{\nobreakdash-\nspace}
\newcommand{\up}{\textsuperscript}
\newcommand{\bslash}{\textbackslash}

\newcommand{\link}[2]{\textcolor{blue}{\href{#1}{#2}}}

\newcommand{\Wrapped}[2][c]{%
  \begin{tabular}[#1]{@{}c@{}}#2\end{tabular}}

\setbeamertemplate{enumerate subitem}{(\roman{enumii})}


\usepackage{wasysym}
\usepackage{marvosym}

\title[Числовые типы данных]{Архитектура компьютеров\texorpdfstring{\\}{ }Лекция 9. Уровень набора инструкций:\texorpdfstring{\\}{}числовые типы данных}


\newcommand{\LRA}{\ensuremath{\leftrightarrow}}
%\newcommand{\m}[1]{\texttt{#1}}
%%%%%%%%%%%%%%%%%%%%%%%%%%%% НАЧАЛО ДОКУМЕНТА %%%%%%%%%%%%%%%%%%%%%%%%%%%%
\begin{document}

\begin{frame}
\titlepage
\end{frame}

\begin{frame}
\frametitle{Содержание}
\tableofcontents[hideallsubsections]
\end{frame}

\begin{frame}{Что значит поддержка числового типа данных}

\pause
\begin{block}{1. Двоичное представление}
    Как будет представлено каждое число из интересующего множества
    в двоичном виде (его \emph{битовая маска}).
\end{block}

\pause
\begin{block}{2. Алгоритмы для арифметических операций}
Для каждой $n$-арной операции и любых $n$ допустимых битовых масок
должен быть определён результат или явно указано на его отсутствие.
Результат должен определяться конструктивно (с помощью алгоритма).
\end{block}

\pause
\begin{block}{Основные задачи}
\pause
\begin{itemize}[<+->]
    \item Эффективность представления и алгоритмов.
    \item Обработка ошибок.
\end{itemize}
\end{block}

\end{frame}

\section{Целые числа со знаком}

\begin{frame}{Базовые константы и операции}

\begin{columns}
    \column{3cm}
\begin{block}{Константы}
\pause
\begin{itemize}
    \item 1,
    \item $-1$,
    \item нуль,
    \item min / max.
\end{itemize}
\end{block}

\pause
    \column{7cm}
\begin{block}{Операции}
\begin{itemize}
    \item {\color{red}Противоположный элемент.}
    \item Сложение: как беззнаковое\\ \hfill(если повезёт).
    \item Вычитание: противоположный и сложение (если повезёт).
\end{itemize}
\end{block}
\end{columns}

\end{frame}

\begin{frame}{Основные способы представления (1)}

\pause
\begin{block}{Знаковый бит (signed magnitude)}
    \begin{itemize}
        \pause \item $-1_{10}$ = \pause $10\ldots 01_2$;
        \pause \item нуль: \pause две штуки, $*0\ldots 0$;
        \pause \item $m$ разрядов $\Rightarrow$ $|\max| = |\min| = $ \pause $2^{m-1} - 1$ = $*1\ldots1_2$;
        \pause \item $1_{10} = 0 \ldots 01_2 \LRA \text{\fbox{поменять старший бит}} \LRA 10 \ldots 01_2 = -1_{10}$.
    \end{itemize}
\end{block}

\pause

\begin{block}{Дополнительный код (two's complement)}
    \begin{itemize}
        \pause \item $-1_{10}$ = \pause $1\ldots 1_2$
            \pause ( = \color{red}{$-2^{m-1} + \sum^{m-2}_{i=0}2^i$} \color{black})
            \pause $\Longleftarrow$  $2^m - 1 = \sum^{m-1}_{i=0}2^i$;
        \pause \item нуль: \pause одна штука, $0\ldots 0$;
        \pause \item $\min$ = \pause $-2^{m-1}$;
            \pause $\max$ = \pause $2^{m-1} - 1$;
        \pause \item $1_{10}$ = $0…01$ → \pause \fbox{not} → $1…10$ → \fbox{inc} → $1…1$ = $-1$.
    \end{itemize}
\end{block}
\end{frame}

\begin{frame}{Основные способы представления (2)}

\begin{block}{$m$-разрядная система со смещением}
\pause
Пример. Пусть $m=4$, смещение $7$ тогда
    \begin{itemize}
        \item  $\min$ = \pause $0000_2$ \pause = $-7_{10}$ (= величина смещения),
        \pause\item  $0_{10}$ = \pause $0111_2$,
        \pause\item  $1_{10}$ = \pause $1000_2$,
        \pause\item  $-1_{10}$ = \pause $0110_2$.
    \end{itemize}
\end{block}

\pause
\emph{Замечание}. Для $m$-разрядных чисел смещение обычно задаётся равным $2^{m-1}$
или $2^{m-1} - 1$.
\end{frame}


\begin{frame}
\frametitle{Некоторые операции со знаковыми целыми}
\begin{itemize}
    \item Вычитание через сложение и взятие противоположного?
        \begin{itemize}
            \pause\item Знаковый бит \pause{\LARGE  \Frowny{}}; %  $-$;
            \pause\item дополнительный код\pause:
            $(3_{10}=0011)$ $+$ $(-2_{10}=1110)$ \pause $= 0001 = 1_{10}$ \pause{\LARGE  \Smiley{}}.  %$+$.
        \end{itemize}
        \pause Следующие операции­ — для дополнительного кода.
    \pause\item Арифметический и логический сдвиги.
    \pause\item Знаковое расширение (sign extension). \\
        \pause Применение:
        \begin{itemize}
            \pause\item деление с помощью \texttt{CWD} / \texttt{IDIV};
            \pause\item преобразования типов \texttt{short} → \texttt{int} в C.
        \end{itemize}
\end{itemize}
\end{frame}

\section{Числа с плавающей точкой}

\begin{frame}
\frametitle{Представления рациональных чисел}
\begin{itemize}
    \pause\item Числа с фиксированной точкой:\\
    \begin{center}
    fix. $n$, $m$ $\Rightarrow$
    $\pm a_n a_{n-1} \ldots a_1 , b_1 \ldots b_{m-1} b_m$.
	\end{center}

    \pause\item Числа с плавающей точкой (знак, мантисса, экспонента):
    \[
        a = \pm f * b^{e}, 0 \leqslant f < 1, b\in \{2,10, \ldots\} ;
    \]

    \pause\item нормализованная мантисса\pause, оптимизация для бинарного случая.
\end{itemize}

\pause
\begin{block}{Пример: мантисса — 3 разряда, экспонента — 2, $b=10$}

    \screenshotw{12cm}{floating-point-vs-real-line.png}

\end{block}


\end{frame}

\begin{frame}
\frametitle{IEEE 754, 1985 г., William Kahan}

\begin{columns}
    \column{7cm}

Стандарт определяет
\begin{itemize}[<+->]
    \item арифметические форматы (числа, $\pm 0$, $\pm \infty$) и способ их кодирования: %\pause
\setbeamertemplate{enumerate item}{\arabic{enumi})}

    \begin{enumerate}
        \item одинарная точность (single precision, 32 бита) aka float,
        \item двойная точность (double precision, 64 бита) aka double,
        \item extended precision (80 бита) aka long double;
    \end{enumerate}
    \item правила выполнения арифметических операций;
    \item правила округления;
    \item поведение в исключительных ситуациях (/ ноль, переполнение…);
    \item рекомендации…
\end{itemize}
%\screenshotw{10cm}{ieee-formats.png}

    \column{5cm}
    \screenshotw{5cm}{Kahan.jpg}

\end{columns}


\end{frame}

\begin{frame}[plain]
\frametitle{Характеристики основных форматов чисел в IEEE 754}
\vspace{-.3cm}\screenshotw{12cm}{ieee-754-props.png}\vspace{-.4cm}
\pause\small В общем случае:
\begin{itemize}
    \item \vspace{-.1cm} в поле экспоненты хранится число, которое нужно уменьшить на~127~(1023), чтобы догадаться о значении экспоненты;
    \pause\item \vspace{-.1cm} в поле мантиссы хранится число $f$ из интервала $[1,2)$ в двоичном виде ($f = f_0 2^0 + f_1 2^{-1} + f_2 2^{-2} \ldots$)\pause, \\ для $\exp\not = \min$ старший бит $f_0$ не хранится, а подразумевается (нормализация+оптимизация).
\end{itemize}
\end{frame}

\begin{frame}
\frametitle{Примеры перевода дробных чисел\\в формат IEEE 754 (одинарная точность)}
\begin{itemize}\itemsep=.5cm
    \item $1.0$ = \pause $+ 1.0 \cdot 2^0$ = \pause
        \\$\qquad$
            sgn: \m{0}, exp: \m{01111111}, m: \m{0\ldots 0} = \pause
        \\$\qquad$\m{3F80 0000},
    \item \pause $0.5$ = \pause
        $+ 1.0 \cdot 2^{-1}$ = \pause
        \\$\qquad$
            sgn: \m{0}, exp: \m{01111110}, m: \m{0\ldots 0} = \pause
        \\$\qquad$\m{3F00 0000},
    \item \pause $1.5$ = \pause
        $+ (1 + 1\cdot2^{-1} ) \cdot 2^0$ =\pause
        \\$\qquad$
            sgn: \m{0}, exp: \m{01111111}, m: \m{10\ldots 0} = \pause
        \\$\qquad$\m{3FC0 0000}.
\end{itemize}
\end{frame}

\begin{frame}
\frametitle{Специальные типы чисел в IEEE 754}
\screenshotw{12cm}{ieee-754-number-types.png}
\end{frame}

\end{document}
