\newcommand{\h}{handout,
}

\documentclass[
\h%
ucs,ignorenonframetext,hyperref={pdftex,unicode},xcolor=dvipsnames]{beamer}
\usepackage[utf8x]{inputenc}
\usepackage[russian]{babel}
\usepackage[T2A]{fontenc}
\usepackage{tikz}
\usetikzlibrary{shapes,arrows,positioning,chains,calc}
\usepackage{graphicx}
\usepackage{fixltx2e}
\usepackage{paratype}
\usepackage{booktabs}
\usepackage{xcolor}
\usepackage{pbox}

\newcommand{\screenshot}[1]{
\begin{center}
\includegraphics[width=12cm,keepaspectratio]{./images/#1}
\end{center}
}

\newcommand{\screenshotw}[2]{
\begin{center}
\includegraphics[width=#1,keepaspectratio]{./images/#2}
\end{center}
}

\usecolortheme{crane}
\useoutertheme{infolines}
\setbeamertemplate{navigation symbols}{}

\author[А. М. Пеленицын]{А.~М.~Пеленицын\texorpdfstring{\\}{ }
apel@sfedu.ru}

\date{Весна 2016}

\institute[Мехмат ЮФУ]{Южный федеральный университет \texorpdfstring{\\}{ }
Институт математики, механики и компьютерных наук им. И.\,И.~Воровича\texorpdfstring{\\}{ }
Кафедра информатики и вычислительного эксперимента}

\subtitle{}

\AtBeginSection[]
{
\begin{frame}<beamer>
\frametitle{Содержание}
\tableofcontents[currentsection,hideothersubsections]
\end{frame}
}

\AtBeginSubsection[]
{
\begin{frame}<beamer>
\frametitle{Содержание}
\tableofcontents[currentsection,subsectionstyle=show/shaded/hide]
\end{frame}
}

\newcommand{\nspace}{\hspace{0pt}}
\newcommand{\nbdash}{\nobreakdash-\nspace}
\newcommand{\up}{\textsuperscript}
\newcommand{\bslash}{\textbackslash}

\newcommand{\link}[2]{\textcolor{blue}{\href{#1}{#2}}}

\newcommand{\Wrapped}[2][c]{%
  \begin{tabular}[#1]{@{}c@{}}#2\end{tabular}}


\title[Коды Хэмминга. Подсистема I/O]{Архитектура компьютеров\texorpdfstring{\\}{ }Лекция 7. Коды Хэмминга.\texorpdfstring{\\}{ }Подсистема ввода-вывода}

\begin{document}

\begin{frame}
\titlepage
\end{frame}

\section{Коды Хэмминга}[fragile]

\begin{frame}{Минимальное кодовое расстояние $d$,
(количество~исправляемых ошибок $t$)}

\begin{columns}
    \column{5.5cm}
\begin{block}{Код 3-кратного повторения: $k=1$, $n=3$\onslide<3->{\pause, $d=3$}}
%\onslide<1->{
\screenshotw{3cm}{Repetition-3-code-geometry.pdf}
%}
\end{block}

    \column{5.5cm}
\onslide<4->{
\begin{block}{Код проверки чётности: $k=2$, $n=3$\onslide<5->{, $d=2$}}
\screenshotw{3cm}{Parity-check-geometry.pdf}
\end{block}
}
\end{columns}
\end{frame}

\begin{frame}{Коды Хэмминга}
\begin{itemize}[<+->]

    \item \textbf{Идея:} добавить ряд битов проверки чётности, по
        корректности
        которых после передачи установить позицию ошибки (if any).

    \item
            $(7, 4, 3)$-код Хэмминга ($n = 7$, $k = 4$, $d=3$):

            \screenshotw{3cm}{Hamming_7_4.pdf}

    \item $(2^r - 1, 2^r - r - 1)$-код Хэмминга: \\
    \begin{itemize}
        \item позиции кодового слова нумеруются с 1;

        \item инфо-биты вписываются подряд в позиции
            кодового слова, пропуская позиции $2^{i-1}$, $1 \leqslant i \leqslant r$;

        \item $i$-й бит проверки чётности ($1 \leqslant i \leqslant r$)
            вписывается на $2^{i-1}$-ю
            позицию кодового слова и контролирует чётность инфо-битов
            в позициях, номер которых содержит $i$-ый бит равный 1.
    \end{itemize}
\end{itemize}
\end{frame}

\begin{frame}{Формулы для $(7,4)$-кода Хэмминга (часть 1)}

\begin{block}{Структура кодового слова $c$}
Пусть исходное сообщение обозначается $d=(d_1, d_2, d_3, d_4)$, а биты проверки чётности — $p_i$, тогда:
\[
    c = (p_1, p_2, d_1, p_3, d_2, d_3, d_4).
\]
\end{block}

\begin{block}{Формулы для $p_i$ через $c_i$}
\begin{enumerate}
    \item $p_1 = c_3 + c_5 + c_7$,
        потому что $3 = 1 \color{red} 1$, \ $5 = 1 0 \color{red} 1$,
            $7 = 1 1 \color{red} 1$;
    \item $p_2 = c_3 + c_6 + c_7 $,
        потому что $3 = \textcolor{red}{1} 1$, \ $6 = 1\textcolor{red}{1}0$,
            $7 = 1\textcolor{red}{1}1$;
    \item $p_3 = c_5 + c_6 + c_7 $,
        потому что $5 = \textcolor{red}{1} 01$, $6 = \textcolor{red}{1}10$,
            $7 = \textcolor{red}{1}11$.
\end{enumerate}
\end{block}
\end{frame}

\begin{frame}{Формулы для $(7,4)$-кода Хэмминга (часть 2)}

\begin{block}{Формулы для $p_i$ через $d_i$}
В приведённых формулах для $p_i$ можно заменить $c_i$ на биты исходного сообщения $d_i$, используя указанную выше структуру кодового слова $c$:
\begin{enumerate}
    \item $p_1 = d_1 + d_2 + d_4$,
    \item $p_2 = d_1 + d_3 + d_4$,
    \item $p_3 = d_2 + d_3 + d_4$.
\end{enumerate}
\end{block}

\begin{block}{Итоговая формула для кодового слова}
\[
    c = (
        \textcolor{green}{d_1 + d_2 + d_4},
        \textcolor{blue}{d_1 + d_3 + d_4},
        d_1,
        \textcolor{magenta}{d_2 + d_3 + d_4},
        d_2,
        d_3,
        d_4).
\]

\textbf{Замечание:} вектор $c$ получен из вектора $d$ как результат \emph{линейного преобразования}.
\end{block}

\end{frame}

\begin{frame}{Проверочная матрица $(7,4)$-кода Хэмминга}
\small
\begin{block}{Крестики превращаются в единички}
Составим таблицу: какие биты проверки чётности отвечают за какие биты кодового слова. На основе этой таблицы запишем \emph{проверочную матрицу} $\mathbf{H}$.

%\begin{table}
\begin{tabular}{cp{6cm}}
%{\scriptsize
    \begin{tabular}{r|cccccccc}
    \toprule
       & $p_1$ & $p_2$ & $d_1$ & $p_3$ & $d_2$ & $d_3$ & $d_4$ \\
    \midrule
    $p_1$ & x  &    & x  &    & x  &    & x \\
    $p_2$ &    & x  & x  &    &    & x  & x \\
    $p_3$ &    &    &    & x  & x  & x  & x \\
    \bottomrule
    \end{tabular}
&
    %\[
    $
        \mathbf{H} := \begin{pmatrix}
            1 & 0 & 1 & 0 & 1 & 0 & 1\\
            0 & 1 & 1 & 0 & 0 & 1 & 1\\
            0 & 0 & 0 & 1 & 1 & 1 & 1
        \end{pmatrix}
    %\]
    $
\end{tabular}%}
%\end{table}
\end{block}

\begin{block}{Основное свойство проверочной матрицы}
Пусть $v$ — принятое по каналу слово,  которое отличается от кодового слова $c$ $(7,4)$-кода Хэмминга не более, чем в одной позиции, тогда вектор (он называется \emph{синдромом})
\[
    s = \mathbf{H}v
\]
даёт двоичную запись номера этой позиции. \color{red}{Если $s=0$, то $v=c$}.
\end{block}

\end{frame}

\section{Ввод-вывод}

\begin{frame}
\frametitle{Логическая организация I/O}
\screenshotw{10cm}{pc-logic-structure.png}
\pause
\begin{block}{Роль контроллера}
\begin{itemize}
    \item \textit{для устройства}: тайминг, контроль;
    \item \textit{для процессора}: декодирование команд, пересылка данных, отчёт о статусе, распознавание адреса, буферизация, контроль ошибок.
\end{itemize}
\end{block}
\end{frame}

\begin{frame}{Логическая структура контроллера}
\screenshotw{12cm}{controller-structure.pdf}
\end{frame}

\begin{frame}{Стили взаимодействия с I/O}
\begin{center}
\renewcommand{\arraystretch}{1.6}
\begin{tabular}{>{\raggedright}p{2cm}|cc}
                       & Без прерываний & С прерываниями  \\ \hline
C участием процессора & Программируемый I/0 & I/O на основе П. \\
Без участия процессора & & Прямой доступ в память\\
\end{tabular}
% (DMA, Direct Memory Access)
\end{center}
\end{frame}

\begin{frame}[plain]{I/O с точки зрения процессора (чтение, 3 стиля)}
\vspace{-.2cm}
\screenshotw{11.5cm}{io-styles.pdf}
\end{frame}

\begin{frame}{I/O с точки зрения программиста}

\pause
\begin{itemize}[<+->]\itemsep=.7cm
    \item Доступ через «порты» (Port-Mapped I/O)\\Пример: инструкции \texttt{IN}/\texttt{OUT} в x86.

    \item Через память (Memory Mapped I/O, MMIO)\\
    Пример: «обычный» \texttt{MOV}.

    \item Программные прерывания\\
    Пример: \texttt{INT} в x86. \\ \pause (Другие типы прерываний: аппаратные, исключения.)
\end{itemize}
\end{frame}

\section{Системные шины}

\begin{frame}
\frametitle{Эволюция шин на ПК}
\begin{itemize}\itemsep=.6cm
    \item ISA (Industry Standard Architecture) @ IBM PC, 1981;
    \item EISA для 32-разрядных IBM-совместимых компьютеров, 1988;
    \item PCI, 1992;
    \item PCIe (Express), 2004.
\end{itemize}
\end{frame}

\begin{frame}
\frametitle{Устройство типичного ПК 1990-х}
\screenshotw{10cm}{pci-isa.png}
\end{frame}

\begin{frame}
\frametitle{Звёздная топология PCIe}
\screenshotw{11cm}{pcie-topology.pdf}
\end{frame}

\begin{frame}
\frametitle{Дорожки (lane) PCIe}
\screenshotw{9cm}{pcie-lane.pdf}
\end{frame}

\begin{frame}
\frametitle{Слоты PCIe и PCI}
\screenshotw{9cm}{pcie-slots.jpg}

•~PCIe~×4, \hfill •~PCIe~×16, \hfill •~PCIe~×1, \hfill •~PCIe~×16, \hfill •~PCI~(32-bit)
\end{frame}

\begin{frame}
\frametitle{Intel motherboard architecture}
\begin{columns}
    \column{6.3cm} 4 Series  (до 2008 г.) \\
        \screenshotw{6cm}{Intel_4_Series_arch.png}
        %\screenshotw{5cm}{motherboard.pdf}

    \pause \column{6.3cm} 5+ Series  (после 2008 г.) \\
        \screenshotw{6cm}{Intel_5_Series_arch.png}
\end{columns}
\end{frame}

\end{document}

\section{Организация некоторых перифирийных устройств}

\begin{frame}
\frametitle{Устройство жидкокристаллического монитора}
\screenshotw{12cm}{lcd.png}
\end{frame}

\begin{frame}
\frametitle{Устройство лазерного принтера}
\screenshotw{9cm}{laser-printer.png}
\end{frame}
