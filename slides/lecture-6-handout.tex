\newcommand{\h}{handout,}

\documentclass[
\h%
ucs,ignorenonframetext,hyperref={pdftex,unicode},xcolor=dvipsnames]{beamer}
\usepackage[utf8x]{inputenc}
\usepackage[russian]{babel}
\usepackage[T2A]{fontenc}
\usepackage{tikz}
\usetikzlibrary{shapes,arrows,positioning,chains,calc}
\usepackage{graphicx}
\usepackage{fixltx2e}
\usepackage{paratype}
\usepackage{booktabs}
\usepackage{xcolor}
\usepackage{pbox}

\newcommand{\screenshot}[1]{
\begin{center}
\includegraphics[width=12cm,keepaspectratio]{./images/#1}
\end{center}
}

\newcommand{\screenshotw}[2]{
\begin{center}
\includegraphics[width=#1,keepaspectratio]{./images/#2}
\end{center}
}

\usecolortheme{crane}
\useoutertheme{infolines}
\setbeamertemplate{navigation symbols}{}

\author[А. М. Пеленицын]{А.~М.~Пеленицын\texorpdfstring{\\}{ }
apel@sfedu.ru}

\date{Весна 2016}

\institute[Мехмат ЮФУ]{Южный федеральный университет \texorpdfstring{\\}{ }
Институт математики, механики и компьютерных наук им. И.\,И.~Воровича\texorpdfstring{\\}{ }
Кафедра информатики и вычислительного эксперимента}

\subtitle{}

\AtBeginSection[]
{
\begin{frame}<beamer>
\frametitle{Содержание}
\tableofcontents[currentsection,hideothersubsections]
\end{frame}
}

\AtBeginSubsection[]
{
\begin{frame}<beamer>
\frametitle{Содержание}
\tableofcontents[currentsection,subsectionstyle=show/shaded/hide]
\end{frame}
}

\newcommand{\nspace}{\hspace{0pt}}
\newcommand{\nbdash}{\nobreakdash-\nspace}
\newcommand{\up}{\textsuperscript}
\newcommand{\bslash}{\textbackslash}

\newcommand{\link}[2]{\textcolor{blue}{\href{#1}{#2}}}

\newcommand{\Wrapped}[2][c]{%
  \begin{tabular}[#1]{@{}c@{}}#2\end{tabular}}


\title[Вторичная память. Подсистема I/O (1)]{Архитектура компьютеров\texorpdfstring{\\}{ }Лекция 6. Вторичная память.\texorpdfstring{\\}{ }Подсистема ввода-вывода}

\begin{document}

\begin{frame}
\titlepage
\end{frame}

%\begin{frame}
%\frametitle{Содержание}
%\tableofcontents[hideallsubsections]
%\end{frame}

\begin{frame}
\frametitle{Иерархия памяти}
\vspace{-1cm}
\begin{columns}
    \column{7cm}
        \screenshotw{9cm}{memory-hierarchy.png}
        %\screenshotw{5cm}{motherboard.pdf}

    \pause \column{4.5cm}
        Рост
        \begin{itemize}
            \item стоимости за 1 бит \pause $\uparrow$,\pause
            \item объёма \pause $\downarrow$,\pause
            \item скорости доступа \pause $\uparrow$.
        \end{itemize}
\end{columns}

\vspace{.5cm}
\pause Энергозависимость.
\end{frame}

\section{Устройство магнитных дисков}

\begin{frame}
\frametitle{Винчестер: модель 1894 года с патронами .30-30}
\screenshotw{12.3cm}{winchester.jpg}
\end{frame}

\begin{frame}
\frametitle{Винчестер IBM 3340 @ System/370, 1973 г.}
Проектировался для хранения двух блоков по 30 Мб.
\screenshotw{10cm}{IBM_magnetic_disk_drives.png}
\pause Первые HDD изобретены IBM в 1956 г.
\end{frame}

\begin{frame}
\frametitle{Схема современного жёсткого диска}
\screenshotw{10cm}{hard-drive.pdf}
\end{frame}

\begin{frame}
\frametitle{Технология магнитной записи}
\screenshotw{12cm}{hdd-head.pdf}
\end{frame}

\begin{frame}
\frametitle{Сектора и дорожки}
    \screenshotw{9cm}{hdd-tracks.pdf}
\end{frame}

\begin{frame}
\frametitle{Фрагмент дорожки жёсткого диска}
\screenshotw{12cm}{hdd-track.png}
\end{frame}

\begin{frame}
\frametitle{Параметры жёсткого диска}
\begin{itemize}[<+->]
    \item скорость позиционирования головки (seek): 5-10 мс;
    \item 5'000–10'000 дорожек;
    \item скорость вращения шпинделя: 5'400, 7'200, 10'800 об./мин.;
    \item 1–12 магнитных пластин;
    \item геометрия диска: цилиндр — головка — сектор (CHS).
\end{itemize}
\end{frame}

%\begin{frame}
%\frametitle{Разбиение жёсткого диска на зоны}
%\screenshotw{7cm}{hdd-zones.png}
%\end{frame}

\begin{frame}
\frametitle{Контроллер жёсткого диска: функции}
\begin{itemize}
    \item Обеспечение простого интерфейса:
    \begin{itemize}
        \item поддержка команд READ, WRITE, FORMAT и т. д.
        \item LBA (Logical Block Addressing, линейный адрес) $\to$ CHS;
    \end{itemize}
    \item преобразование последовательности битов в байты и наоборот;
    \item кэширование;
    \item учёт повреждённых секторов.
\end{itemize}
\end{frame}

\begin{frame}{SSD — «убийца» HDD?}
\screenshotw{11cm}{ssd-img.jpg}
\end{frame}

\section[Интерфейсы вторичной памяти и некоторых I/O-устройств]{Интерфейсы вторичной памяти и некоторых устройств ввода-вывода}

\begin{frame}
\frametitle{Жёсткие диски IDE (Integrated Drive Electronic)}
aka ATA (AT Attachment, AT = Advanced Technology от IBM PC AT)
\screenshotw{10cm}{ide-ribbon.pdf}
\begin{itemize}[<+->]
    \item середина 1980-х, IBM PC XT, BIOS-интерфейс;
    \item CHS: 10-4-6 бит $\mapsto$ 504 Мб;
    \item EIDE (Extended IDE / ATA-2):
    \begin{itemize}
        \item 28 бит для LBA, заменившего CHS  $\mapsto$ 128 Гб;
        \item до 4-х дисков; CD-ROM / DVD;
    \end{itemize}
    \item ATAPI (Packet Interface, более «умная» шина);
    \item 2003 г. Serial ATA (SATA).
\end{itemize}
\end{frame}

\begin{frame}
\frametitle{SCSI (Small Computer System Interface), 1979/86 г.}
\screenshotw{3.5cm}{scsi_logo.pdf}
\begin{itemize}
    \item “SCSI is an intelligent, peripheral, buffered, peer to peer interface”;
    \item интерфейс «премиум-класса»\\(компьютеры Sun SPARC, Apple в 90-е);
    \item топология ёлочной гирлянды;
    \item параллельный интерфейс.
\end{itemize}
\end{frame}

\begin{frame}[plain]{Загружаем Windows 95 по SCSI}
\vspace{-.3cm}
\screenshotw{11cm}{boot-list-win95-scsi-c.png}
\end{frame}

\begin{frame}
\frametitle{RAID\\(redundant array of independent/inexpensive disks)}
\begin{itemize}[<+->]
    \item 1988 г., статья Д. Паттерсона и др.;
    \item RAID vs. SLED (Single Large Expensive Disk);
    \item параллелизм;
    \item шесть оригинальных «уровней»: RAID 0, RAID 1, …
    \item серверные решения, SCSI.
\end{itemize}
\end{frame}

\begin{frame}
\frametitle{RAID 0, 1, 10}
\screenshotw{11cm}{raid.png}
\end{frame}

\section {Контроль ошибок памяти}


\begin{frame}
\frametitle{ECC — Error-Correcting/Controlling Codes (помехоустойчивые коды) (1)}

\begin{itemize}%[<+->]
    \onslide<1->\item Сообщения длины $k$ и кодовые слова длины $n$, $n > k$;

    \onslide<2->\item метрика Хэмминга;

    \onslide<4->\item простейший код, контролирующий ошибки:
    \begin{itemize}
        \item код проверки чётности;
    \end{itemize}

    \onslide<6->\item простейший код, исправляющий ошибки:
    \begin{itemize}
        \item код троекратного повторения.
    \end{itemize}
\end{itemize}
\end{frame}

\begin{frame}[fragile]
\frametitle{ECC — Error-Correcting/Controlling Codes (2)}

\small
\begin{columns}
    \column{6.5cm}
    Кодовое пространство $\mathbb F_2^3$
        ($\mathbb F_2 = \{0,1\}$):

\screenshotw{3cm}{F_3_2-geometry.pdf}

Вычисление расстояния Хэмминга:

\screenshotw{3cm}{Hamming_distance-example.pdf}

    \column{5.9cm}
Код проверки чётности:
\screenshotw{3cm}{Parity-check-geometry.pdf}

Код троекратного повторения:
\screenshotw{3cm}{Repetition-3-code-geometry.pdf}
\end{columns}

\end{frame}

\end{document}

%%%%%%%%%%% END OF DOCUMENT %%%%%%%%%%%%%%%%%%%%%%%%%%
