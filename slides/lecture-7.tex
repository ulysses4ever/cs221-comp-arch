\newcommand{\h}{handout,
}

\documentclass[
\h%
ucs,ignorenonframetext,hyperref={pdftex,unicode},xcolor=dvipsnames]{beamer}
\usepackage[utf8x]{inputenc}
\usepackage[russian]{babel}
\usepackage[T2A]{fontenc}
\usepackage{tikz}
\usetikzlibrary{calc,trees,positioning,arrows,chains,shapes.geometric,%
    decorations.pathreplacing,decorations.pathmorphing,shapes,%
    matrix,shapes.symbols}
\usepackage{graphicx}
\usepackage{fixltx2e}
\usepackage{paratype}
\usepackage{booktabs}
\usepackage{xcolor}
\usepackage{pbox}

\usepackage[font=itshape]{quoting}

\newcommand{\screenshot}[1]{
\begin{center}
\includegraphics[width=12cm,keepaspectratio]{./images/#1}
\end{center}
}

\newcommand{\screenshotw}[2]{
\begin{center}
\includegraphics[width=#1,keepaspectratio]{./images/#2}
\end{center}
}

\usecolortheme{crane}
\useoutertheme{infolines}
\setbeamertemplate{navigation symbols}{}

\author[А. М. Пеленицын]{А.~М.~Пеленицын\texorpdfstring{\\}{ }
apel@sfedu.ru}

\date{Весна 2016}

\institute[Мехмат ЮФУ]{Южный федеральный университет \texorpdfstring{\\}{ }
Институт математики, механики и компьютерных наук им. И.\,И.~Воровича\texorpdfstring{\\}{ }
Кафедра информатики и вычислительного эксперимента}

\subtitle{}

\AtBeginSection[]
{
\begin{frame}<beamer>
\frametitle{Содержание}
\tableofcontents[currentsection,hideothersubsections]
\end{frame}
}

\AtBeginSubsection[]
{
\begin{frame}<beamer>
\frametitle{Содержание}
\tableofcontents[currentsection,subsectionstyle=show/shaded/hide]
\end{frame}
}

\newcommand{\nspace}{\hspace{0pt}}
\newcommand{\nbdash}{\nobreakdash-\nspace}
\newcommand{\up}{\textsuperscript}
\newcommand{\bslash}{\textbackslash}

\newcommand{\link}[2]{\textcolor{blue}{\href{#1}{#2}}}

\newcommand{\Wrapped}[2][c]{%
  \begin{tabular}[#1]{@{}c@{}}#2\end{tabular}}

\setbeamertemplate{enumerate subitem}{(\roman{enumii})}


\title[Коды Хэмминга. Подсистема I/O (2)]{Архитектура компьютеров\texorpdfstring{\\}{ }Лекция 7. Коды Хэмминга.\texorpdfstring{\\}{ }Подсистема ввода-вывода (окончание)}

\begin{document}

\begin{frame}
\titlepage
\end{frame}

\section{Коды Хэмминга}[fragile]

\begin{frame}{Коды Хэмминга, $d=3$}
\begin{itemize}[<+->]

    \item Идея: добавить ряд битов проверки чётности, по корректности
        которых после передачи установить позицию ошибки (if any)

    \item
            $(7, 4)$-код Хэмминга ($n = 7$, $k = 4$):

\only<1-6>{\screenshotw{3cm}{Hamming_7_4.pdf}}
\only<7>{%
            \vspace{.5cm}

\begin{tabular}{cp{6cm}}
{\scriptsize
\begin{tabular}{rcccccccc}
   & p1 & p2 & d1 & p4 & d2 & d3 & d4 \\
p1 & x  &    & x  &    & x  &    & x \\
p2 &    & x  & x  &    &    & x  & x \\
p4 &    &    &    & x  & x  & x  & x \\
\end{tabular}}
&
            %\[
            $
            \mathbf{H} := \begin{pmatrix}
1 & 0 & 1 & 0 & 1 & 0 & 1\\
0 & 1 & 1 & 0 & 0 & 1 & 1\\
0 & 0 & 0 & 1 & 1 & 1 & 1
            \end{pmatrix}
            %\]%
            $
\end{tabular}

            \vspace{.625cm}
}

    \item $(2^r - 1, 2^r - r - 1)$-код Хэмминга: \\
    \begin{itemize}
        \item позиции кодового слова нумеруются с 1;

        \item инфо-биты вписываются подряд в позиции
            кодового слова, пропуская позиции $2^i$, $0 \leqslant i < r$;

        \item $i$-й бит проверки чётности вписывается на $2^i$-ю
            позицию кодового слова и контролирует чётность инфо-битов
            в позициях, номер которых содержит $i$-ый бит равный 1.
    \end{itemize}
\end{itemize}
\end{frame}

\begin{frame}{$(15,11)$-код Хэмминга}
\screenshotw{11cm}{Hamming_15_11.png}

\pause
Проверочная матрица $(15,11)$-кода Хэмминга:
\[
\left( \begin{array}{ccccccccccccccc}
1 & 0 & 1 & 0 & 1 & 0 &1 & 0 &1 & 0 &1 & 0 &1 & 0 &1 \\
0 & 1 & 1 & 0 &0 & 1 & 1 & 0 &0 & 1 & 1 & 0 &0 & 1 & 1\\
0 & 0 & 0 & 1 &1 & 1 & 1 & 0 & 0 & 0 & 0 & 1 &1 & 1 & 1\\
0 & 0 & 0 & 0 &0 & 0 & 0 & 1 &1 & 1 & 1 & 1 &1 & 1 & 1
\end{array} \right)
\]
\end{frame}

\section{Ввод-вывод и системные шины}

\begin{frame}
\frametitle{Логическая организация I/O}
\screenshotw{10cm}{pc-logic-structure.png}

\pause

\begin{columns}
    \column{6cm}
\begin{itemize}[<+->]
    \item «Программируемый I/O» или Port-Mapped I/O (пример: инструкции \texttt{IN}/\texttt{OUT} в x86);
    \item Memory Mapped I/O (MMIO);
    \item Direct Memory Access (DMA);
\end{itemize}

    \column{6.5cm}
\begin{itemize}[<+->]
    \item прерывания: программные (пример: \texttt{INT} в x86) и аппаратные; исключения;
    \item арбитр шины и «кража циклов».
\end{itemize}
\end{columns}

\begin{itemize}[<+->]
    \item Роль контроллера;
    \item «программируемый I/O» или Port-Mapped I/O (пример: инструкции \texttt{IN}/\texttt{OUT} в x86);
    \item Memory Mapped I/O (MMIO);
    \item Direct Memory Access (DMA);
    \item прерывания: программные (пример: \texttt{INT} в x86) и аппаратные; исключения;
    \item арбитр шины и «кража циклов».
\end{itemize}
\end{frame}

\begin{frame}
\frametitle{Эволюция шин на ПК}
\begin{itemize}
    \item ISA (Industry Standard Architecture) @ IBM PC, 1981;
    \item EISA для 32-разрядных IBM-совместимых компьютеров, 1988;
    \item PCI, 1992;
    \item PCIe (Express), 2004.
\end{itemize}
\end{frame}

\begin{frame}
\frametitle{Peripheral Component Interconnect (Conventional PCI)}
\screenshotw{8cm}{pci-slots.jpg}
\end{frame}

\begin{frame}
\frametitle{Устройство типичного ПК 1990-х}
\screenshotw{10cm}{pci-isa.png}
\end{frame}

\begin{frame}
\frametitle{Звёздная топология PCIe}
\screenshotw{10cm}{PCIe-sample-arch.pdf}
\end{frame}

\begin{frame}
\frametitle{Intel motherboard architecture}
\begin{columns}
    \column{6.3cm} 4 Series  (до 2008 г.) \\
        \screenshotw{6cm}{Intel_4_Series_arch.png}
        %\screenshotw{5cm}{motherboard.pdf}

    \pause \column{6.3cm} 5+ Series  (после 2008 г.) \\
        \screenshotw{6cm}{Intel_5_Series_arch.png}
\end{columns}
\end{frame}

\section{Организация некоторых перифирийных устройств}

\begin{frame}
\frametitle{Устройство жидкокристаллического монитора}
\screenshotw{12cm}{lcd.png}
\end{frame}

\begin{frame}
\frametitle{Устройство лазерного принтера}
\screenshotw{9cm}{laser-printer.png}
\end{frame}

\end{document}
