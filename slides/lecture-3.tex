\newcommand{\h}{handout,%
}

\documentclass[
\h%
ucs,ignorenonframetext,hyperref={pdftex,unicode},xcolor=dvipsnames]{beamer}
\usepackage[utf8x]{inputenc}
\usepackage[russian]{babel}
\usepackage[T2A]{fontenc}
\usepackage{tikz}
\usetikzlibrary{shapes,arrows,positioning,chains,calc}
\usepackage{graphicx}
\usepackage{fixltx2e}
\usepackage{paratype}
\usepackage{booktabs}
\usepackage{xcolor}
\usepackage{pbox}

\newcommand{\screenshot}[1]{
\begin{center}
\includegraphics[width=12cm,keepaspectratio]{./images/#1}
\end{center}
}

\newcommand{\screenshotw}[2]{
\begin{center}
\includegraphics[width=#1,keepaspectratio]{./images/#2}
\end{center}
}

\usecolortheme{crane}
\useoutertheme{infolines}
\setbeamertemplate{navigation symbols}{}

\author[А. М. Пеленицын]{А.~М.~Пеленицын\texorpdfstring{\\}{ }
apel@sfedu.ru}

\date{Весна 2016}

\institute[Мехмат ЮФУ]{Южный федеральный университет \texorpdfstring{\\}{ }
Институт математики, механики и компьютерных наук им. И.\,И.~Воровича\texorpdfstring{\\}{ }
Кафедра информатики и вычислительного эксперимента}

\subtitle{}

\AtBeginSection[]
{
\begin{frame}<beamer>
\frametitle{Содержание}
\tableofcontents[currentsection,hideothersubsections]
\end{frame}
}

\AtBeginSubsection[]
{
\begin{frame}<beamer>
\frametitle{Содержание}
\tableofcontents[currentsection,subsectionstyle=show/shaded/hide]
\end{frame}
}

\newcommand{\nspace}{\hspace{0pt}}
\newcommand{\nbdash}{\nobreakdash-\nspace}
\newcommand{\up}{\textsuperscript}
\newcommand{\bslash}{\textbackslash}

\newcommand{\link}[2]{\textcolor{blue}{\href{#1}{#2}}}

\newcommand{\Wrapped}[2][c]{%
  \begin{tabular}[#1]{@{}c@{}}#2\end{tabular}}


\title[Архитектура компьютеров. Лекция 3]{Архитектура компьютеров\texorpdfstring{\\}{ }Лекция 3. Разнообразие компьютеров}

\begin{document}

\begin{frame}
\titlepage
\end{frame}

\begin{frame}[plain]{Закон Мура (Гордон Мур, Дэвид Хаус, 1965)}
\vspace{-.2cm}\screenshotw{11.2cm}{moores_law2.pdf}
\end{frame}

\begin{frame}
\frametitle{Содержание}
\tableofcontents
\end{frame}

\section {«Одноразовые компьютеры»}

\begin{frame}{RFID (Radio-frequency identification) метки}
\begin{columns}
    \column{3cm} \screenshotw{3cm}{rfid_rice.jpg}
    \column{5cm} \screenshotw{5cm}{RFID_Tags.jpg}
\end{columns}
\pause
\begin{itemize}
    \item RFID vs. штрих-коды
\end{itemize}
\end{frame}

\begin{frame}{RFID (II)}
    \vspace{-.48cm}
\begin{columns}
    \column{6cm} \screenshotw{5.5cm}{rfid_transponder.jpg}

    \pause\screenshotw{4cm}{rfid-sheep.jpg}
    \column{5cm}

    \pause\screenshotw{3.8cm}{rfid-home.jpg}
\end{columns}
\end{frame}

\begin{frame}{RFID (III)}
\screenshotw{5cm}{biom-passport.jpg}
\end{frame}

\begin{frame}{Биометрические паспорта}
\screenshotw{6cm}{rfid_passport.jpg}
\end{frame}

\section {Микроконтроллеры (встраиваемые компьютеры)}

\begin{frame}
\frametitle{Характерные черты (отличия от настольных ПК)}
\begin{itemize}[<+->]
    \item Гарвардская архитектура (ROM для программы, RAM для данных),
    \item интеграция ЦП / памяти / IO,
    \item ценовые ограничения,
    \item физические ограничения,
    \item системы реального времени.
\end{itemize}
\end{frame}

\begin{frame}
\frametitle{Arduino (single-board microcontroller) @ Atmel AVR}
\screenshotw{2cm}{arduino-logo.png}
\screenshotw{6cm}{Arduino.jpg}
\end{frame}

\begin{frame}
\frametitle{Arduino IDE (Wiring, C++)}
\screenshotw{6cm}{Arduino-IDE.png}
\end{frame}

\section {Игровые приставки}

\begin{frame}
\frametitle{Примеры и характерные черты}
%\screenshotw{12cm}{game-consoles-1.png}
\begin{columns}
    \column{6cm}
        \screenshotw{6cm}{PS4_s.png}

    \column{6cm}
        \screenshotw{6cm}{Xbox_One_s.png}
\end{columns}
\pause
\link%
    {http://en.wikipedia.org/wiki/History_of_video_game_consoles_(eighth_generation)}%
    {Wikipedia: History of video game consoles (eighth generation)}

\pause\begin{itemize}[<+->]
    \item Переход с RISC-процессоров (Power PC) на CISC-процессоры AMD,
    \item высокая степень параллелизма,
    \item продвинутое I/O,
    \item не расширяемы,
    \item + портативные системы, смартфоны, планшеты, Smart TV~— RISC-процессоры (ARM).
\end{itemize}

\end{frame}

\section {Персональные компьютеры}

\begin{frame}
\frametitle{Больше, чем desktop}
\begin{itemize}
    \item Ноутбуки,
    \item нетбуки,
    \item «ультрабуки».
\end{itemize}
\end{frame}

\begin{frame}
\frametitle{Семейство процессоров Intel x86}
\screenshotw{11cm}{intel-family.png}
\end{frame}

\section {Серверы}

\begin{frame}
\frametitle{Сервер Мехмата}
\screenshotw{5.2cm}{mmcs-server.jpg}
\end{frame}

\section {Высокопроизводительные вычисления (HPC)}

\begin{frame}
\frametitle{Виды высокопроизводительных систем}
\begin{itemize}
    \item Компьютерные кластеры
    \item Мейнфреймы
    \item Суперкомпьютеры
\end{itemize}
\end{frame}

\begin{frame}
\frametitle{Типичный кластер под управлением Linux}
\screenshotw{9.5cm}{cluster.jpg}
\pause Серверная ферма, Datacenter (ЦОД), Грид-системы
\end{frame}

\begin{frame}
\frametitle{Мейнфрейм IBM System Z10}
\screenshotw{3.4cm}{Systemz10.jpg}
\end{frame}

\begin{frame}
\frametitle{Суперкомпьютер Cray-2}
\screenshotw{10cm}{Cray2.jpg}
\end{frame}

\begin{frame}
\frametitle{Суперкомпьютер IBM Blue Gene}
\screenshotw{11cm}{blue-gene.jpg}
\end{frame}

\begin{frame}
\frametitle{Суперкомпьютер Ломоносов (МГУ, T-Платформы)}
\screenshotw{11cm}{Lomonosov.png}
\end{frame}

\begin{frame}
\frametitle{Рейтинг cуперкомпьютеров Top500 (\#46, Ноябрь 2015)}
%\small LINPACK
\screenshotw{12cm}{top500-2015.png}
\end{frame}

\end{document}

