\newcommand{\h}{%
handout,%
}

\documentclass[
\h%
ucs,ignorenonframetext,hyperref={pdftex,unicode},xcolor=dvipsnames]{beamer}
\usepackage[utf8x]{inputenc}
\usepackage[russian]{babel}
\usepackage[T2A]{fontenc}
\usepackage{tikz}
\usetikzlibrary{shapes,arrows,positioning,chains,calc}
\usepackage{graphicx}
\usepackage{fixltx2e}
\usepackage{paratype}
\usepackage{booktabs}
\usepackage{xcolor}
\usepackage{pbox}

\newcommand{\screenshot}[1]{
\begin{center}
\includegraphics[width=12cm,keepaspectratio]{./images/#1}
\end{center}
}

\newcommand{\screenshotw}[2]{
\begin{center}
\includegraphics[width=#1,keepaspectratio]{./images/#2}
\end{center}
}

\usecolortheme{crane}
\useoutertheme{infolines}
\setbeamertemplate{navigation symbols}{}

\author[А. М. Пеленицын]{А.~М.~Пеленицын\texorpdfstring{\\}{ }
apel@sfedu.ru}

\date{Весна 2016}

\institute[Мехмат ЮФУ]{Южный федеральный университет \texorpdfstring{\\}{ }
Институт математики, механики и компьютерных наук им. И.\,И.~Воровича\texorpdfstring{\\}{ }
Кафедра информатики и вычислительного эксперимента}

\subtitle{}

\AtBeginSection[]
{
\begin{frame}<beamer>
\frametitle{Содержание}
\tableofcontents[currentsection,hideothersubsections]
\end{frame}
}

\AtBeginSubsection[]
{
\begin{frame}<beamer>
\frametitle{Содержание}
\tableofcontents[currentsection,subsectionstyle=show/shaded/hide]
\end{frame}
}

\newcommand{\nspace}{\hspace{0pt}}
\newcommand{\nbdash}{\nobreakdash-\nspace}
\newcommand{\up}{\textsuperscript}
\newcommand{\bslash}{\textbackslash}

\newcommand{\link}[2]{\textcolor{blue}{\href{#1}{#2}}}

\newcommand{\Wrapped}[2][c]{%
  \begin{tabular}[#1]{@{}c@{}}#2\end{tabular}}


\title[Уровень набора инструкций]%
    {Архитектура компьютеров\texorpdfstring{\\}{ }%
    Лекция 8. Уровень набора инструкций (ISA)}

\begin{document}

\begin{frame}
\titlepage
\end{frame}

\section{Введение}

\begin{frame}{Современные многоуровневые машины}
%\screenshotw{6cm}{}
\centering
{\small
\tikzset{
>=stealth',
  punktchain/.style={
    rectangle,
    rounded corners,
    % fill=black!10,
    draw=black, very thick,
    text width=15em,
    minimum height=2em,
    text centered,
    top color=white,
    on chain},
  every join/.style={->, thick,shorten >=1pt},
}
\begin{tikzpicture}
  [node distance=.7cm,
  start chain=going below,]
     \node[punktchain, join, bottom color=lime!60] (PL)
         {Языки «высокого уровня»};

     \node[punktchain, join, bottom color=teal!75] (ASM)
         {Язык ассемблера};

     \node[punktchain, join, bottom color=blue!50] (BIN)
         {«Машинный код»\\(уровень набора инструкций, Instruction Set Arch., \textbf{ISA})};

     \node[punktchain, join, bottom color=yellow!60] (MuCode)
         {Микрокод процессора\\(\textbf{микроархитектура})};

     \node[punktchain, join, bottom color=red!60] (DL)
         {Схемы цифровой логики};
\end{tikzpicture}
}

\end{frame}

\begin{frame}
\frametitle{Место уровня ISA}
\screenshotw{11cm}{isa-place.png}
\end{frame}

%\begin{frame}
%\frametitle{Содержание}
%\tableofcontents[hideallsubsections]
%\end{frame}

\begin{frame}{Определение}

\pause
\begin{block}{Википедия (en)}
Набор инструкций (\textbf{instruction set architecture, ISA}) это часть архитектуры компьютера, \alert{относящаяся к программированию}:
\begin{columns}
    \column{6cm}
\begin{itemize}
    \item встроенные (native) типы данных,
    \item инструкции,
    \item регистры,
    \item режимы адресации,
\end{itemize}
    \column{6cm}
\begin{itemize}
    \item архитектуру памяти,
    \item механизм обработки прерываний (исключений),
    \item правила взаимодействия с I/O.
\end{itemize}
\end{columns}
\end{block}

\pause
\begin{block}{Замечательное наблюдение}
Первые электронные компьютеры имели простые наборы инструкций~— так же, как и современные (\alert{RISC}).
\end{block}
\end{frame}

\section{Набор инструкций MIPS}

\begin{frame}{MIPS: общая характеристика}
\begin{itemize}[<+->]
    \item Разработан в Стэнфорде в начале 80-х;
    \item автор: Джон Хеннесси и др., см. книгу Паттерсона \& Хеннесси;
    \item коммерциализация: MIPS Technologies (Imagination Technologies);
    \item рынок: встраиваемые устройства (игровые консоли),
    в середине-конце 90-х каждый третий проданный RISC-процессор это MIPS-процессор;
    \item \link{http://stackoverflow.com/q/2635086/465100}{MIPS processors: Are they still in use?} / StackOverflow.
\end{itemize}
\end{frame}

\begin{frame}{\link{https://geektimes.ru/post/273192/}{Baikal-T1} (2015) @ MIPS32 r5 от «Байкал Электроникс»}
\screenshotw{9.5cm}{baikal-t1.jpg}
\end{frame}

\begin{frame}[fragile]{Инструкции и регистры}

\begin{columns}
    \column{6.1cm}
\pause
\begin{block}{Группы инструкций}
\begin{itemize}
    \item Арифметические,
    \item логические,
    \item перемещение данных,
    \item условные и безусловные переходы.
\end{itemize}
\end{block}

\pause

\begin{block}{«Адресность» инструкций}
Пример: \texttt{a = b + c} переводится в
\begin{verbatim}
    add a, b, c
\end{verbatim}
\end{block}


\pause
    \column{4.5cm}
%Операнды (часть 1)
%\screenshotw{5.7cm}{MIPS-operands.png}
\begin{block}{Регистры (32 штуки)}
\begin{itemize}[<+->]
    \item \m{\$s0}, …, \m{\$s7},
    \item \m{\$t0}, …, \m{\$t9},
    \item \m{\$a0}, …, \m{\$a3},
    \item \m{\$v0}, \m{\$v1},
    \item \m{\$ra},
    \item \m{\$sp}, \m{\$gp}, \m{\$fp},
    \item \m{\$zero},
    \item \m{\$at}.
\end{itemize}
\end{block}
\end{columns}
\end{frame}

\begin{frame}[fragile]{Интеллектуальная игра: откомпилируй C в MIPS}

\pause
\begin{columns}
    \column{5.5cm}
\begin{block}{Код на C}
\begin{verbatim}
f = (g + h) - (i + j);\end{verbatim}
\end{block}

    \column{5.5cm}
\begin{block}{Код на ассемблере MIPS}
Предполагая, что \texttt{f}, …, \texttt{j} находятся в \texttt{\$s0}, …, \texttt{\$s4}:
\pause
\begin{verbatim}
    add $t0, $s1, $s2
    add $t1, $s3, $s4
    sub $s0, $t0, $t1
\end{verbatim}
\end{block}
\end{columns}
\end{frame}

\begin{frame}{Модель памяти}
\begin{columns}
    \column{6.5cm}
\pause
\begin{itemize}[<+->]
    \item Память адресуется в байтах.
    \item Требование по выравниванию (адрес слова кратен 4).
    \item Составные типы данных размещаются в памяти.
    \item \emph{Переключаемый порядок} байт
    (bi-endian).
    \item Load-Store Architecture.
\end{itemize}
\pause
    \column{5cm}
\screenshotw{5cm}{MIPS-mem-model.pdf}
\end{columns}

\end{frame}

\begin{frame}[fragile]{Режимы адресации (продолжение игры)}
\begin{columns}
    \column{5.5cm}
\begin{block}{Код на C}
\begin{verbatim}
    A[12] = h + A[8];\end{verbatim}
\end{block}

    \column{5.5cm}
\onslide<3>{%
\begin{block}{Непосредственные операнды}
%\begin{verbatim}
   \texttt{addi \$s2, \$s1, -1}
%\end{verbatim}
\end{block}
}
\end{columns}

\begin{columns}
    \column{9cm}

\begin{block}{Код на ассемблере MIPS}
Предполагаем, что \texttt{h} находится в \texttt{\$s2}, и\\
адрес начала массива \texttt{A} — в \texttt{\$s3}.
\pause
\begin{verbatim}
    lw  $t0, 32($s3)    # load word
    add $t0, $s2, $t0
    sw  $t0, 48($s3)    # store word\end{verbatim}
\end{block}
    \column{2cm}
\end{columns}
\end{frame}

\begin{frame}{Код инструкций («машинный код»)}
%\screenshotw{12.3cm}{MIPS-ins-format.png}
%\begin{columns}

%    \column{6.3cm}

%    \column{5cm}
%\end{columns}
%\begin{table}[]
%\centering
%\caption{My caption}
%\label{my-label}
{\footnotesize\renewcommand{\arraystretch}{1.3}
\begin{tabular}{l|cccccc|l}
\hline
\multicolumn{1}{c}{\begin{tabular}[c]{@{}c@{}}Имя\\ формата\end{tabular}} & \multicolumn{6}{c}{Поля}                                           & Комментарий                          \\ \hline
                                                                          & 6 бит  & 5 бит & 5 бит & 5 бит       & 5 бит       & 6 бит         & = 32 бита  \\
R-формат                                                                  & opcode & rs    & rt    & rd          & shamt       & функция       & Арифметика            \\
I-формат                                                                  & opcode & rs    & rt    & \multicolumn{3}{c|}{адрес/непоср. операнд} & \begin{tabular}[c]{@{}l@{}}Ветвления +\\ непоср.оп.\end{tabular} \\
J-формат                                                                  & opcode & \multicolumn{5}{c|}{адрес для перехода (jump)}             & Безусловные п. \\ \hline
\end{tabular}}
%\end{table}

\pause
\begin{block}{Коды («номера») регистров}
\begin{columns}\small
    \column{4cm}
\begin{itemize}
    \item \texttt{\$t0–\$t7} $\mapsto$  8–15;
\end{itemize}
    \column{4cm}
\begin{itemize}
    \item \texttt{\$s0–\$s7} $\mapsto$ 16–23;
\end{itemize}
    \column{4cm}
\begin{itemize}
    \item \texttt{\$t8–\$t9} $\mapsto$ 24–25.
\end{itemize}
\end{columns}
\end{block}


\pause
\begin{columns}
    \column{9.5cm}
\begin{block}{Пример: \texttt{add \$t0, \$s1, \$s2}}
\small
\ttfamily
\begin{tabular}{|c|c|c|c|c|c|}
\hline
special & \$s1 & \$s2 & \$t0 & 0 & add\\
\hline
0 & 17 & 18 & 8 & 0 & 32\\
\hline
000000 & 10001 & 10010 & 01000 & 00000 & 100000\\
\hline
\end{tabular}
\end{block}
    \column{1cm}
\end{columns}


\end{frame}

\begin{frame}[fragile]{Переходы}

\begin{columns}

    \column{5.5cm}
\pause
\begin{block}{\small Условные (branch)}
\begin{itemize}\ttfamily\small
    \item beq rs, rt, L1
    \item bne rs, rt, L1
\end{itemize}
\end{block}

\pause\vspace{-.1cm}
\begin{block}{\small Сравнения}
\begin{itemize}\ttfamily\small
    \item slt rd, rs, rt
    \item slti rt, rs, constant
\end{itemize}
\end{block}

\pause\vspace{-.1cm}
\begin{block}{\small Безусловные (jump)}
\begin{itemize}\ttfamily\small
    \item j L1
    \item jal ProcedureAddress
    \item jr \$ra
\end{itemize}
\end{block}

    \column{5.5cm}
\pause
\begin{block}{Код на C}
\begin{verbatim}if (i==j) f = g+h;
else f = g-h;\end{verbatim}
\end{block}

\pause
\begin{block}{Код на ассемблере MIPS}
\begin{verbatim}
      bne $s3, $s4, Else
      add $s0, $s1, $s2
      j   Exit
Else: sub $s0, $s1, $s2
Exit: …\end{verbatim}
\end{block}
\end{columns}
\end{frame}

\begin{frame}[plain]{Режимы адресации MIPS: резюме}
\vspace{-.1cm}
\screenshotw{7cm}{MIPS-addressing-modes.png}
\end{frame}

\end{document}
