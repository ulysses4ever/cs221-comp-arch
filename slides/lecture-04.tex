\newcommand{\h}{handout,%
}

\documentclass[
\h%
ucs,ignorenonframetext,hyperref={pdftex,unicode},xcolor=dvipsnames]{beamer}
\usepackage[utf8x]{inputenc}
\usepackage[russian]{babel}
\usepackage[T2A]{fontenc}
\usepackage{tikz}
\usetikzlibrary{shapes,arrows,positioning,chains,calc}
\usepackage{graphicx}
\usepackage{fixltx2e}
\usepackage{paratype}
\usepackage{booktabs}
\usepackage{xcolor}
\usepackage{pbox}

\newcommand{\screenshot}[1]{
\begin{center}
\includegraphics[width=12cm,keepaspectratio]{./images/#1}
\end{center}
}

\newcommand{\screenshotw}[2]{
\begin{center}
\includegraphics[width=#1,keepaspectratio]{./images/#2}
\end{center}
}

\usecolortheme{crane}
\useoutertheme{infolines}
\setbeamertemplate{navigation symbols}{}

\author[А. М. Пеленицын]{А.~М.~Пеленицын\texorpdfstring{\\}{ }
apel@sfedu.ru}

\date{Весна 2016}

\institute[Мехмат ЮФУ]{Южный федеральный университет \texorpdfstring{\\}{ }
Институт математики, механики и компьютерных наук им. И.\,И.~Воровича\texorpdfstring{\\}{ }
Кафедра информатики и вычислительного эксперимента}

\subtitle{}

\AtBeginSection[]
{
\begin{frame}<beamer>
\frametitle{Содержание}
\tableofcontents[currentsection,hideothersubsections]
\end{frame}
}

\AtBeginSubsection[]
{
\begin{frame}<beamer>
\frametitle{Содержание}
\tableofcontents[currentsection,subsectionstyle=show/shaded/hide]
\end{frame}
}

\newcommand{\nspace}{\hspace{0pt}}
\newcommand{\nbdash}{\nobreakdash-\nspace}
\newcommand{\up}{\textsuperscript}
\newcommand{\bslash}{\textbackslash}

\newcommand{\link}[2]{\textcolor{blue}{\href{#1}{#2}}}

\newcommand{\Wrapped}[2][c]{%
  \begin{tabular}[#1]{@{}c@{}}#2\end{tabular}}


\title[Центральный процессор]{Архитектура компьютеров\texorpdfstring{\\}{ }Лекция 4. Центральный процессор}

\begin{document}

\begin{frame}
\titlepage
\end{frame}

\begin{frame}
\frametitle{Содержание}
\tableofcontents%[hideallsubsections]
\end{frame}

\section {Центральный процессор}

\subsection{Организация и порядок работы}

\begin{frame}
\frametitle{Схема компьютера с общей шиной}
\screenshotw{10cm}{one-cpu.png}
\end{frame}

\begin{frame}
\frametitle{АЛУ + регистры + шины = тракт данных (Datapath)}
\screenshotw{8cm}{data-path.png}
\end{frame}

\begin{frame}
\frametitle{УУ: выполнение инструкции}
\begin{enumerate}[<+->]
    \item Загрузка очередной инструкции из памяти в регистр IR\\
    (адрес этой инструкции хранится в регистре PC).
    \item ++PC
    \item Определение типа инструкции (декодирование).
    \item Определение местонахождения операндов.\\
        При необходимости загрузка операндов из памяти.
    \item Выполнение инструкции.
    \item Запись результата.
    \item goto 1.
\end{enumerate}
\end{frame}

\subsection{Микропрограммное управление}

\begin{frame}{Software vs. Hardware}
\begin{itemize}\small
\item Аппаратное обеспечение: платы, интегральные схемы, кабели…
\item Программное обеспечение: алгоритмы, их запись на конкретных ЯП.
\end{itemize}

\pause\begin{block}{Основной Тезис}
Аппаратное и программное обеспечение логически эквивалентны.
\end{block}

\pause{\small\begin{block}{Карен Панетта Ленц}
“Hardware is just petrified software.”
\end{block}}

{\small\pause\begin{block}{Дискриминирующие факторы}
\pause\begin{itemize}
\addtolength{\itemsep}{-0.3\baselineskip}
    \item Стоимость,
    \item быстродействие,
    \item надёжность,
    \item частота изменений.
\end{itemize}
\end{block}}
\end{frame}

\begin{frame}{Изобретение микропрограммирования}
\begin{columns}
    \column{6cm}
\begin{itemize}[<+->]\itemsep=.7cm
    \item 1940-е: только ISA и ЦЛУ;
    \item 1951 г., Морис~Уилкс (Maurice~Wilkes), университет~Кэмбриджа: идея трёхуровнего компьютера, интерпретация команд ISA;
    \item 1970-е: идея микропрограммирования укрепилась в мейнстриме.
\end{itemize}

    \column{5.5cm}
\onslide<2->{\screenshotw{5.5cm}{Wilkes.jpg}}
\end{columns}
\end{frame}

\begin{frame}{1960/70-е: гонка наборов инструкций}
\begin{itemize}\itemsep=.4cm
\item из нашего опыта: INC vs. ADD 1 \pause(и не только);
\pause\item микропрограммные реализации:
    \begin{itemize}\itemsep=.2cm
    \item умножение/деление,
    \item плавающая точка,
    \item вызов подпрограмм,
    \item циклы,
    \item обработка символьных строк,
    \item etc. (индексация, релокация, прерывания, переключение процессов, аудио/видео…).
    \end{itemize}
\end{itemize}
\end{frame}

\begin{frame}
\frametitle{Преимущества интерпретации команд}
\begin{block}{Простота}\pause
\begin{itemize}\itemsep=.4cm
    \item отладки и документирования,
    \item исправления,
    \item добавления.
\end{itemize}
\end{block}
\end{frame}

\begin{frame}
\frametitle{Компьютеры с интерпретаторами: CISC}

\begin{block}{\alert{Complex} Instruction Set Computer: 60-е — 80-е гг.}
\begin{itemize}\itemsep=.4cm
    \item Компьютеры IBM,
    \item все компьютеры 60-х/70-х, кроме самых дорогих (CDC),
    \item персональные компьютеры,
    \item роль ПЗУ (ROM).
\end{itemize}
\end{block}
\end{frame}

\begin{frame}
\frametitle{CISC vs. RISC: 90-е — 2010-е гг.}
\begin{itemize}\itemsep=.6cm
    \item RISC-архитектуры (\textbf{\textcolor{blue!60!black}{Reduced}} Instruction Set Computer):
    \begin{itemize}\itemsep=.2cm
        \item RISC I / II в Беркли (Дэвид Паттерсон, Карло Секвин; 1980),\\Sun SPARC,
        \item MIPS в Стенфорде (Джон Хеннеси; Sony PS, PS 2, PSP),
        \item IBM Power Architecture (Cell в PS 3, маршрутизаторы Cisco),
        \item ARM (Android-гаджеты, IPad, MS Surface, Canon PowerShot A470, …),
        \item Atmel AVR (Arduino).
    \end{itemize}
    \pause\item CISC (монополисты): Мейнфреймы IBM, x86, DEC VAX.
    \pause\item Проблемы RISC, завоевание новых рынков.
\end{itemize}
\end{frame}

%Современные процессоры и параллелизм.
\section {Принципы проектирования современных процессоров}
\begin{frame}
\frametitle{Принципы проектирования современных процессоров / принципы RISC}
\pause
\begin{itemize}
    \item Отражают состояние технологии.
\end{itemize}

\pause{\hrule\begin{itemize}[<+->]
    \item Аппаратное исполнение,
    \item высокая скорость запуска инструкций на выполнение (MIPS),
    \item лёгкость декодирования инструкций,
    \item обращение к памяти производится только\\
        специальными инструкциями (“Load-Store Architectures”),
    \item регистров должно быть много.
\end{itemize}}
\end{frame}

\section {Организация параллельных вычислений}

\subsection {Параллелизм на уровне команд}

\begin{frame}
\frametitle{Конвейер (pipeline)}
\begin{columns}
    \column{7cm}
        \begin{itemize}
            \item Начало: 1959, IBM, prefetch buffer.
        \end{itemize}

    \column{6cm}
        \begin{itemize}
            \item
            \onslide<5->{\small\link{http://en.wikipedia.org/wiki/Classic_RISC_pipeline}{Wikipedia: Classic RISC pipeline}}
        \end{itemize}
\end{columns}

\pause{\screenshotw{11cm}{pipeline-1.png}}
\pause{\screenshotw{7cm}{pipeline-2.png}}
\pause{\begin{itemize}
    \item mips vs. latency
\end{itemize}}
\end{frame}

\begin{frame}
\frametitle{Два конвейера}
\screenshotw{11cm}{two-pipelines.png}
\pause{\begin{itemize}[<+->]
    \item Pentium: u- и v-конвейер.
    \item Больше?
\end{itemize}}
\end{frame}

\begin{frame}
\frametitle{Суперскалярная архитектура процессора}
\screenshotw{9.5cm}{superscalar.png}
\pause{\vspace{-.5cm}\begin{itemize}[<+->]
    \item Начало: CDC 6600.
    \item Вопрос: {\small что позволило уместить все блоки C4 на одном процессоре?}
    \item SISD~— Single Instruction, Single Data в \small{<<\alert{таксономии Флинна}>> (Flynn)}.
\end{itemize}}

\end{frame}

\subsection {Параллелизм на уровне процессоров}

\begin{frame}{Матричные и векторные процессоры — SIMD}
\screenshotw{9cm}{array-processor.png}

\pause
\begin{itemize}
    \item Векторный процессор $\Longleftarrow$ векторные регистры.
    \item \pause Современные примеры: \pause MMX/SSE, \pause Cell, \pause (GP)GPU: NVIDIA CUDA.
\end{itemize}


\end{frame}


\begin{frame}{Мультипроцессоры и мультикомпьютеры —~\small{MIMD~(NUMA)}}
\screenshotw{6cm}{multiproc-1.png}
\pause{\hrule\screenshotw{6cm}{multiproc-2.png}}
\end{frame}

\end{document}
