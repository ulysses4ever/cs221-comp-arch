\newcommand{\h}{handout,%
}

\documentclass[
\h%
ucs,ignorenonframetext,hyperref={pdftex,unicode},xcolor=dvipsnames]{beamer}
\usepackage[utf8x]{inputenc}
\usepackage[russian]{babel}
\usepackage[T2A]{fontenc}
\usepackage{tikz}
\usetikzlibrary{calc,trees,positioning,arrows,chains,shapes.geometric,%
    decorations.pathreplacing,decorations.pathmorphing,shapes,%
    matrix,shapes.symbols}
\usepackage{graphicx}
\usepackage{fixltx2e}
\usepackage{paratype}
\usepackage{booktabs}
\usepackage{xcolor}
\usepackage{pbox}

\usepackage[font=itshape]{quoting}

\newcommand{\screenshot}[1]{
\begin{center}
\includegraphics[width=12cm,keepaspectratio]{./images/#1}
\end{center}
}

\newcommand{\screenshotw}[2]{
\begin{center}
\includegraphics[width=#1,keepaspectratio]{./images/#2}
\end{center}
}

\usecolortheme{crane}
\useoutertheme{infolines}
\setbeamertemplate{navigation symbols}{}

\author[А. М. Пеленицын]{А.~М.~Пеленицын\texorpdfstring{\\}{ }
apel@sfedu.ru}

\date{Весна 2016}

\institute[Мехмат ЮФУ]{Южный федеральный университет \texorpdfstring{\\}{ }
Институт математики, механики и компьютерных наук им. И.\,И.~Воровича\texorpdfstring{\\}{ }
Кафедра информатики и вычислительного эксперимента}

\subtitle{}

\AtBeginSection[]
{
\begin{frame}<beamer>
\frametitle{Содержание}
\tableofcontents[currentsection,hideothersubsections]
\end{frame}
}

\AtBeginSubsection[]
{
\begin{frame}<beamer>
\frametitle{Содержание}
\tableofcontents[currentsection,subsectionstyle=show/shaded/hide]
\end{frame}
}

\newcommand{\nspace}{\hspace{0pt}}
\newcommand{\nbdash}{\nobreakdash-\nspace}
\newcommand{\up}{\textsuperscript}
\newcommand{\bslash}{\textbackslash}

\newcommand{\link}[2]{\textcolor{blue}{\href{#1}{#2}}}

\newcommand{\Wrapped}[2][c]{%
  \begin{tabular}[#1]{@{}c@{}}#2\end{tabular}}

\setbeamertemplate{enumerate subitem}{(\roman{enumii})}


\title[Архитектура компьютеров. Лекция 1]{Архитектура компьютеров\texorpdfstring{\\}{ }Лекция 1. Введение}

\begin{document}

\begin{frame}
\titlepage
\end{frame}

\begin{frame}
\frametitle{Содержание}
\tableofcontents[hideallsubsections]
\end{frame}

\section {Организационная информация}

\begin{frame}{Всё, что вы хотели знать о курсе\ldots}
\begin{enumerate}
	\pause\item Продолжительность = 1 семестр (2 модуля), отчётность = зачёт.

	\pause\item БАРС (подробно~---
		в \link{http://edu.mmcs.sfedu.ru/mod/resource/view.php?id=5977}{УКД}):{\footnotesize
		\begin{itemize}
		\item зачёт ${\geqslant}$ 60 баллов,
		\item ТК/РК = 59/41,
		\item ТК: лаб, дз-лаб, дз-лек (доработка лаб — в рамках дз-лаб!).
		\item РК: 2 ${\times}$ КР-лек = 15+10, КР-лаб = 16.
		\end{itemize}}

	\pause\item Дедлайны для дз (все публикуются воскресенье вечером): {\footnotesize
		\begin{itemize}
		\item д/з по лекциям  = 1 неделя (но лучше делать до новой лекции),
		\item д/з по практике = 11 дней.
		\end{itemize}}

	\pause\item Страница курса «\link{http://edu.mmcs.sfedu.ru/course/view.php?id=170}{Архитектура компьютеров}» на \link{http://edu.mmcs.sfedu.ru}{edu.mmcs.sfedu.ru}.{\footnotesize
		\begin{enumerate}
		\item Регистрация на сайте.
		\item Запись на курс с ключом: три цифры ABC, A =  номер курса (2), B = номер группы, C = левая или правая половинка в расписании (1/2) (кроме 2.3 и 2.4, у них — 23 и 24).
		\end{enumerate}}

	\pause\item \link{http://mmcs.sfedu.ru/~ulysses/timetable.html}{Моё расписание},
		почта — на первом слайде,
        \link{http://forum.mmcs.sfedu.ru/t/2-kurs-pmi-i-fiit-cs221-arhitektura-kompyutera/831}{форум}.
		\\ \footnotesize{Консультации (ауд. 203): пт (15:20), сб (15:20) или на переменах}
\end{enumerate}
\end{frame}

\begin{frame}{Литература}
\begin{enumerate}
	\item \emph{Таненбаум~Э., Остин~Т.} Архитектура компьютера / \textbf{6-е изд.}(+CD) — СПб.: Питер, 2013. — 816 с.

	\item \emph{Stallings~W.} Computer Organization and Architecture /  Prentice Hall; 9 edition (March 11, 2012). 792 p.

    \item \emph{Паттерсон Д., Хеннесси Д.} Архитектура компьютера и проектирование компьютерных систем / \textbf{4-е изд.} — СПб.: Питер, 2012. — 784 с.

	\item \emph{Орлов~С.\,П., Ефимушкина~Н.\,В.}
	\link{http://window.edu.ru/resource/007/77007/files/organizatsiya.pdf}%
	{Организация компьютерных систем}:
	Учебное пособие / Самара: Самарск. гос. тех. ун-т, 2001. — 203 с.

	\item \link{http://www.it.ru/books/it_history_1.pdf}{Страницы истории отечественных ИТ} / Сост. Э.М. Пройдаков. — М.:
	Альпина Паблишер, 2015.
\end{enumerate}
\end{frame}

\begin{frame}{Домашнее задание}
\begin{itemize}
	\item (Желательно) раздобыть книгу Таненбаума (6-е издание!).
	\item Прочитать по ней начало главы 1 («Введение») до пункта «Современные многоуровневые машины» включительно и Приложение В (про ассемблер).
	\item Понимать работу инструкций MUL, DIV и LOOP ассемблера Intel~8088.
	\item «Домашнее задание 1» (лекции) на странице курса — в воскресенье вечером.
\end{itemize}
\end{frame}

\section{Зачем нужен курс архитектуры компьютера}

\begin{frame}{Довод 1: консенсус экспертов}
\begin{columns}
    \column{0cm}
    \column{10cm}
\begin{block}{Чему нужно учить специалистов по информатике}
    \textcolor{blue}{\href{https://www.acm.org/education/CS2013-final-report.pdf}{Computer Science Curricula 2013}}
    \small by
    \begin{itemize}
        \item Association for Computing Machinery (ACM),
        \item IEEE Computer Society.
    \end{itemize}
\end{block}
\end{columns}

\pause
\begin{block}{Фрагмент CSC2013: Architecture and Organization, цитата}
\begin{quoting}
A professional in any field of computing should \textbf{not} regard the computer as just a \textbf{black
box} that executes programs by \textbf{magic}.
\end{quoting}
\end{block}
\end{frame}

\begin{frame}{Довод 2: исследования в институте в области HPC}
High-Perfomance Computing

\begin{block}{Научная школа ВЦ РГУ → ЮГИНФО ЮФУ → Института ММКН}
\textit{Разработка и развитие специальных программных средств для управления вычислительными ресурсами центров высокопроизводительных вычислений.} // \link{http://uginfo.sfedu.ru/science}{uginfo.sfedu.ru/science}
\end{block}
\end{frame}

\begin{frame}{Научная школа кафедры АДМ в области HPC}
\screenshotw{12cm}{ops_seminar_2011.jpg}

— \link{http://ops.rsu.ru/about.shtml}{ops.rsu.ru}
\end{frame}

\begin{frame}{Довод 3: прикладное программирование}
\screenshotw{7cm}{sipser-vs-videogames.jpg}
\end{frame}

\begin{frame}{Довод 3: прикладное программирование}
Проблема 7-классника (ВКШ)
\screenshotw{12cm}{fp-pitfalls.jpg}
\end{frame}


\section{Организация и архитектура вычислительных систем}

\begin{frame}{Организация vs. архитектура}

\pause
\begin{block}{Архитектура (\textasciitilde{} «интерфейс»)}
Атрибуты системы, видимые программисту. Контракт.

\pause
\begin{itemize}
	\item Набор инструкций, которые понимает данный процессор,
	\item количество бит для представления чисел,
	\item способы адресации (указания места операндов инструкций).
\end{itemize}
\end{block}

\pause
\begin{block}{Организация (\textasciitilde{} «реализация»)}
Набор устройств и способ их взаимосвязи для реализации
контракта (архитектуры).

\pause
\begin{itemize}
	\item Напряжение тока на кристалле процессора,
	\item технологии связи устройств (шины),
	\item технологии реализации памяти.
\end{itemize}
\end{block}
\end{frame}

\begin{frame}{Организация vs. архитектура: «жизненные» примеры}

\pause
\begin{block}{«Архитектура X86» (x64, ARM, …)}
\begin{itemize}
    \item Intel,
    \item AMD.
\end{itemize}
\end{block}

\pause
\begin{block}{Микроархитектура процессоров Intel: Haswell (Ivy Bridge, …)}
\begin{itemize}
    \item разный объём кэш-памяти,
    \item наличие дополнительных ускорителей.
\end{itemize}
\end{block}

\end{frame}

\begin{frame}{Организация: основы}

\pause
\begin{columns}
	\column{6cm}
	\screenshotw{6cm}{a-computer.pdf}\pause

	\column{6cm}
	\screenshotw{6cm}{cpu.pdf}
\end{columns}
\end{frame}

\begin{frame}{Современная многоуровневая архитектура}
\pause
\begin{columns}
	\column{6cm}
%vspace{-.2cm}
%\screenshotw{6cm}{multi-level-computer.pdf}
{\small
\tikzset{
>=stealth',
  punktchain/.style={
    rectangle,
    rounded corners,
    % fill=black!10,
    draw=black, very thick,
    text width=15em,
    minimum height=2em,
    text centered,
    top color=white,
    on chain},
  every join/.style={->, thick,shorten >=1pt},
}
\begin{tikzpicture}
  [node distance=.7cm,
  start chain=going below,]
     \node[punktchain, join, bottom color=lime!60] (PL)
         {Языки «высокого уровня»};

     \node[punktchain, join, bottom color=teal!75] (ASM)
         {Язык ассемблера};

     \node[punktchain, join, bottom color=blue!50] (BIN)
         {«Машинный код»\\(уровень набора инструкций, Instruction Set Arch., \textbf{ISA})};

     \node[punktchain, join, bottom color=yellow!60] (MuCode)
         {Микрокод процессора\\(\textbf{микроархитектура})};

     \node[punktchain, join, bottom color=red!60] (DL)
         {Схемы цифровой логики};
\end{tikzpicture}
}

	\column{6.5cm}
\pause
\vspace{-.5cm}
\begin{itemize}[<+->]
	\item компиляторы, статические анализаторы,
			IDE, библиотеки;

	\vspace{.5cm}
	\item (диз)ассемблеры, отладчики, линкёры;

	\vspace{.6cm}
	\item ОЗУ, системная шина, ЦП;

	\vspace{1.5cm}
	\item внутренняя шина, тракт данных;

	\vspace{.9cm}
	\item логические вентили и схемы.
\end{itemize}
\end{columns}
\end{frame}

\begin{frame}{Уровни, скрытые на прошлом слайде}

\large
\begin{itemize}\itemsep=.5cm
    \item Предметно-ориентированные языки\\
    (Domain-Specific Languages, DSL).

    \item Операционная система.

    \item Физический уровень.
\end{itemize}

\end{frame}

\section {Немного о практике}

\begin{frame}[fragile]{Пример (ассемблер as88)}
\begin{columns}
	\column{7.5cm}
\begin{block}{Сложение двух чисел}
\begin{verbatim}
.SECT .TEXT
        MOV     AX, (x)
        ADD     AX, (y)
        MOV     (res), AX

.SECT .DATA
x:      .WORD   37
y:      .WORD   5

.SECT .BSS
res:    .SPACE  2
\end{verbatim}
\end{block}
	\column{4.5cm}\pause
Элементы языка:
\begin{itemize}
	\item мнемоники,
	\item директивы,
	\item метки,
	\item способы адресации.
\end{itemize}
\end{columns}
\end{frame}

\end{document}
