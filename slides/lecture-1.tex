\newcommand{\h}{handout,%
}

\documentclass[
\h%
ucs,ignorenonframetext,hyperref={pdftex,unicode},xcolor=dvipsnames]{beamer}
\usepackage[utf8x]{inputenc}
\usepackage[russian]{babel}
\usepackage[T2A]{fontenc}
\usepackage{tikz}
\usetikzlibrary{calc,trees,positioning,arrows,chains,shapes.geometric,%
    decorations.pathreplacing,decorations.pathmorphing,shapes,%
    matrix,shapes.symbols}
\usepackage{graphicx}
\usepackage{fixltx2e}
\usepackage{paratype}
\usepackage{booktabs}
\usepackage{xcolor}
\usepackage{pbox}

\usepackage[font=itshape]{quoting}

\newcommand{\screenshot}[1]{
\begin{center}
\includegraphics[width=12cm,keepaspectratio]{./images/#1}
\end{center}
}

\newcommand{\screenshotw}[2]{
\begin{center}
\includegraphics[width=#1,keepaspectratio]{./images/#2}
\end{center}
}

\usecolortheme{crane}
\useoutertheme{infolines}
\setbeamertemplate{navigation symbols}{}

\author[А. М. Пеленицын]{А.~М.~Пеленицын\texorpdfstring{\\}{ }
apel@sfedu.ru}

\date{Весна 2016}

\institute[Мехмат ЮФУ]{Южный федеральный университет \texorpdfstring{\\}{ }
Институт математики, механики и компьютерных наук им. И.\,И.~Воровича\texorpdfstring{\\}{ }
Кафедра информатики и вычислительного эксперимента}

\subtitle{}

\AtBeginSection[]
{
\begin{frame}<beamer>
\frametitle{Содержание}
\tableofcontents[currentsection,hideothersubsections]
\end{frame}
}

\AtBeginSubsection[]
{
\begin{frame}<beamer>
\frametitle{Содержание}
\tableofcontents[currentsection,subsectionstyle=show/shaded/hide]
\end{frame}
}

\newcommand{\nspace}{\hspace{0pt}}
\newcommand{\nbdash}{\nobreakdash-\nspace}
\newcommand{\up}{\textsuperscript}
\newcommand{\bslash}{\textbackslash}

\newcommand{\link}[2]{\textcolor{blue}{\href{#1}{#2}}}

\newcommand{\Wrapped}[2][c]{%
  \begin{tabular}[#1]{@{}c@{}}#2\end{tabular}}

\setbeamertemplate{enumerate subitem}{(\roman{enumii})}


\title[Архитектура компьютеров. Лекция 1]{Архитектура компьютеров\texorpdfstring{\\}{ }Лекция 1. Введение}

\begin{document}

\begin{frame}
\titlepage
\end{frame}

\begin{frame}
\frametitle{Содержание}
\tableofcontents[hideallsubsections]
\end{frame}

\section {Организационная информация}

\begin{frame}{Всё, что вы хотели знать о курсе\ldots}
\begin{itemize}[<+->]
	\item Продолжительность = 1 семестр (2 модуля), отчётность = зачёт.
	\item БАРС (подробно~---
		в \link{http://edu.mmcs.sfedu.ru/mod/resource/view.php?id=5977}{УКД}):\footnotesize{
		\begin{itemize}
		\item зачёт ${\geqslant}$ 60 баллов,
		\item ТК/РК = 45/55,
		\item ТК: лаб, дз-лаб, дз-лек; доработка лаб — в рамках дз-лаб!.
		\item РК: 2 ${\times}$ КР-лек = 20+15, КР-лаб = 20.
		\end{itemize}}
	\item Дедлайны для дз (все публикуются в пятницу вечером): \footnotesize{
		\begin{itemize}
		\item д/з по лекциям  = 1 неделя (лучше делать до новой лекции),
		\item д/з по практике = 11 дней.
		\end{itemize}}
	\item Страница курса «\link{http://edu.mmcs.sfedu.ru/course/view.php?id=170}{Архитектура компьютеров}» на \link{http://edu.mmcs.sfedu.ru}{edu.mmcs.sfedu.ru}.\footnotesize{
		\begin{enumerate}
		\item Регистрация на сайте.
		\item Запись на курс с ключом:\\
		три цифры ABC, A =  номер курса (2), B = номер группы,\\
		C = номер недели, на которой вы пришли на практику впервые (1/2).
		\end{enumerate}}
	\item \link{http://mmcs.sfedu.ru/~ulysses/timetable.html}{Моё расписание},
		почта — на первом слайде, форум
		\\ \footnotesize{консультации (ауд. 203): ср (14:00), чт (14:00), сб (16:00) или на переменах}
\end{itemize}
\end{frame}

\begin{frame}{Литература}
\begin{enumerate}
	\item \emph{Таненбаум~Э., Остин~Т.} Архитектура компьютера / 6-е изд.(+CD) — СПб.: Питер, 2013. — 816 с.

	\item \emph{Stallings~W.} Computer Organization and Architecture /  Prentice Hall; 9 edition (March 11, 2012). 792 p.

	\item \emph{Орлов~С.\,П., Ефимушкина~Н.\,В.}
	\link{http://window.edu.ru/resource/007/77007/files/organizatsiya.pdf}%
	{Организация компьютерных систем}:
	Учебное пособие / Самара: Самарск. гос. тех. ун-т, 2001. — 203 с.

	\item \link{http://www.it.ru/books/it_history_1.pdf}{Страницы истории отечественных ИТ} / Сост. Э.М. Пройдаков. — М.:
	Альпина Паблишер, 2015.
\end{enumerate}
\end{frame}

\begin{frame}{Домашнее задание}
\begin{itemize}
	\item (желательно) раздобыть книгу Таненбаума (6-е издание!)
	\item прочитать по ней начало главы 1 («Введение») до пункта «Современные многоуровневые машины» включительно и Приложение В (про ассемблер)
	\item понимать работу инструкций MUL, DIV и LOOP ассемблера Intel~8088
	\item «Домашнее задание 1» (лекции) на странице курса — в пятницу вечером
\end{itemize}
\end{frame}

\section{Организация и архитектура вычислительных систем}

\begin{frame}{Организация vs. архитектура}

\pause
\begin{block}{Архитектура}
Атрибуты системы, видимые программисту. Интерфейс.

\pause
\begin{itemize}
	\item Инструкции, которые понимает данный процессор,
	\item количество бит для представления чисел,
	\item способы адресации.
\end{itemize}
\end{block}

\pause
\begin{block}{Организация}
Набор устройств и способ их взаимосвязи для реализации
спецификации (архитектуры).

\pause
\begin{itemize}
	\item Напряжение тока для передачи сигналов,
	\item интерфейсы для связи устройств,
	\item технологии памяти.
\end{itemize}
\end{block}
\end{frame}

\begin{frame}{Организация: примеры}
\begin{columns}
	\column{6cm}
	\screenshotw{6cm}{a-computer.pdf}\pause

	\column{6cm}
	\screenshotw{6cm}{cpu.pdf}
\end{columns}
\end{frame}

\begin{frame}{Подход к понятию многоуровневой архитектуры}
\begin{itemize}[<+->]
	\item Назначение компьютера,
	\item машинный язык,
	\item проблема несоответствия,
	\item решение: языки и трансляторы
	\item техническое замечание:\\
		трансляторы = компиляторы + интерпретаторы\\
		\onslide<6>{\footnotesize{(в книге Таненбаума: компиляторы = трансляторы).}}
\end{itemize}
\end{frame}

\begin{frame}{Современная многоуровневая архитектура}
\begin{columns}
	\column{6cm}
%vspace{-.2cm}
\screenshotw{6cm}{multi-level-computer.pdf}
	\column{6.5cm}
\pause
Организация:
\begin{itemize}[<+->]
	\item компиляторы, статические анализаторы,
			IDE, библиотеки;

	\vspace{.6cm}
	\item (диз)ассемблеры, отладчики, линкёры;

	\vspace{.6cm}
	\item ОЗУ, системная шина, ЦП,

	\vspace{.8cm}
	\item внутренняя шина, тракт данных

	\vspace{.9cm}
	\item логические вентили и схемы из них, булевы формулы
\end{itemize}
\end{columns}
\end{frame}

\section {Немного о практике}

\begin{frame}[fragile]{Пример (ассемблер as88)}
\begin{columns}
	\column{7.5cm}
\begin{block}{Сложение двух чисел}
\begin{verbatim}
.SECT .TEXT
        MOV     AX, (x)
        ADD     AX, (y)
        MOV     (res), AX

.SECT .DATA
x:      .WORD   37
y:      .WORD   5

.SECT .BSS
res:    .SPACE  2
\end{verbatim}
\end{block}
	\column{4.5cm}\pause
Элементы языка\pause:
\begin{itemize}
	\item мнемоники,
	\item директивы,
	\item метки,
	\item способы адресации.
\end{itemize}
\end{columns}
\end{frame}

\begin{frame}[fragile]{Пример посложней (ассемблер NASM)}
\begin{block}{Длина C-строки}
\footnotesize{
\begin{verbatim}
     ;---------------------------------------------------
     ; Функция zstr_count:
     ; Определяет длину ASCII-строки, завершающейся нулём
     ; вход:   eax = адрес первого символа строки
     ; выход:  ecx = длина строки

1                      zstr_count:
2  00000030 B9FFFFFFFF      mov  ecx, -1
3
4                      .loop:
5  00000035 41              inc  ecx
6  00000036 803C0800        cmp  byte [eax + ecx], 0
7  0000003A 75F9            jne  .loop
8
9  0000003C C3              ret
\end{verbatim}}
\end{block}
\end{frame}

\begin{frame}[fragile]{Утилиты ассемблирования программ для Intel 8088}
\begin{columns}
	\column{5cm}
\begin{itemize}
	\item as88 — ассемблер,
	\item s88 — симулятор,
	\item t88 — трасёр (отладчик).
\end{itemize}
\pause
	\column{6cm}
\begin{block}{Файл \texttt{test.s}}
\begin{verbatim}
as88 test.s
t88 test.s
\end{verbatim}
\end{block}
\end{columns}
\end{frame}

\end{document}
