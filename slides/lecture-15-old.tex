\newcommand{\h}{%
handout,%
}

\documentclass[
\h%
ucs,ignorenonframetext,hyperref={pdftex,unicode},xcolor=dvipsnames]{beamer}
\usepackage[utf8x]{inputenc}
\usepackage[russian]{babel}
\usepackage[T2A]{fontenc}
\usepackage{tikz}
\usetikzlibrary{shapes,arrows,positioning,chains,calc}
\usepackage{graphicx}
\usepackage{fixltx2e}
\usepackage{paratype}
\usepackage{booktabs}
\usepackage{xcolor}
\usepackage{pbox}

\newcommand{\screenshot}[1]{
\begin{center}
\includegraphics[width=12cm,keepaspectratio]{./images/#1}
\end{center}
}

\newcommand{\screenshotw}[2]{
\begin{center}
\includegraphics[width=#1,keepaspectratio]{./images/#2}
\end{center}
}

\usecolortheme{crane}
\useoutertheme{infolines}
\setbeamertemplate{navigation symbols}{}

\author[А. М. Пеленицын]{А.~М.~Пеленицын\texorpdfstring{\\}{ }
apel@sfedu.ru}

\date{Весна 2016}

\institute[Мехмат ЮФУ]{Южный федеральный университет \texorpdfstring{\\}{ }
Институт математики, механики и компьютерных наук им. И.\,И.~Воровича\texorpdfstring{\\}{ }
Кафедра информатики и вычислительного эксперимента}

\subtitle{}

\AtBeginSection[]
{
\begin{frame}<beamer>
\frametitle{Содержание}
\tableofcontents[currentsection,hideothersubsections]
\end{frame}
}

\AtBeginSubsection[]
{
\begin{frame}<beamer>
\frametitle{Содержание}
\tableofcontents[currentsection,subsectionstyle=show/shaded/hide]
\end{frame}
}

\newcommand{\nspace}{\hspace{0pt}}
\newcommand{\nbdash}{\nobreakdash-\nspace}
\newcommand{\up}{\textsuperscript}
\newcommand{\bslash}{\textbackslash}

\newcommand{\link}[2]{\textcolor{blue}{\href{#1}{#2}}}

\newcommand{\Wrapped}[2][c]{%
  \begin{tabular}[#1]{@{}c@{}}#2\end{tabular}}


\title[Компьютерные шины]{Архитектура компьютеров\texorpdfstring{\\}{ }Лекция 10. Цифровой логический уровень:\texorpdfstring{\\}{ }компьютерные шины}


\begin{document}

\begin{frame}
\titlepage
\end{frame}

\begin{frame}
\frametitle{Содержание}
\tableofcontents
\end{frame}

%\screenshotw{3.2cm}{transistor-NOT.png}

\section{Вопросы организации шин}

\begin{frame}
\frametitle{«Системная шина»}
\screenshotw{9cm}{system-bus-ru.png}

\pause Внутренние и внешние шины
\end{frame}

\begin{frame}[plain]
\frametitle{Разнообразие шин}
\screenshotw{11cm}{h87.png}
\end{frame}

\begin{frame}
\frametitle{Цоколёвка (pinout) процессора}
\screenshotw{11cm}{cpu-pinout.png}
\end{frame}

\begin{frame}
\frametitle{Способы подсоединения к шине}
\begin{block}{Задающие (master) и подчинённые (slave)}
\footnotesize
%\screenshotw{9cm}{master-slave.png}
\begin{table}[h]
\begin{tabular}{@{}lll@{}}
\toprule
\multicolumn{1}{c}{\textbf{Задающее устройство}} & \multicolumn{1}{c}{\textbf{Подчинённое устройство}} & \multicolumn{1}{c}{\textbf{Пример запроса}} \\ \midrule
Центральный процессор                            & Память                                              & Чтение очередной команды         \\
Центральный процессор                            & Устройство ввода-вывода                             & Запрос на передачу данных                   \\
Центральный процессор                            & Сопроцессор                                         & Передача команды                            \\
Устройство ввода-вывода                          & Память                                              & Прямой доступ в память                \\
Сопроцессор                                      & Центральный процессор                               & Доступ к операндам из ЦП                    \\ \bottomrule
\end{tabular}
\end{table}

%\vspace{-.3cm}
\end{block}

\pause
\begin{block}{Технические аспекты}
\begin{itemize}[<+->]
    \item Драйвер, приёмник и трансивер шины.
    \item tri-state устройства vs. открытый коллектор («монтажное ИЛИ», wired OR).
\end{itemize}
\end{block}
\end{frame}

\begin{frame}
\frametitle{Основные характеристики шины}
\pause
\begin{block}{Ширина}\pause
\begin{itemize}[<+->]
    \item Parallel vs. serial,
    \item перекос шины,
    \item временн\'{о}е мультиплексирование.
\end{itemize}
\end{block}

\begin{columns}
    \column{5cm}
    \pause

    \begin{block}{Синхронизация}
    \begin{itemize}
        \item Синхронные,
        \item асинхронные,
        \item изохронные шины.
    \end{itemize}
    \end{block}

    \column{5cm}
    \pause

    \begin{block}{Арбитраж}
    \begin{itemize}
        \item Централизованный,
        \item децентрализованный.
    \end{itemize}
    \end{block}

\end{columns}
\end{frame}

\begin{block}{Проблемы эволюции шин}
\screenshotw{10cm}{addresses-evolution-x86.png}
\end{block}

\subsection{Синхронизация шины}

\begin{frame}
\frametitle{Чтение на синхронной шине}
\screenshotw{11cm}{sync-bus-transfer.png}
\end{frame}

\begin{frame}
\frametitle{Чтение на асинхронной шине}
\screenshotw{8.5cm}{async-bus-transfer.png}

\pause
\small\begin{block}{Полное квитирование (full handshaking)}
\vspace{-.2cm}
\begin{enumerate}
  \setlength{\itemsep}{3pt}
  \setlength{\parskip}{0pt}
  \setlength{\parsep}{0pt}
    \item Установка MSYN.
    \item Установка SSYN.
    \item Сброс MSYN в ответ на установку SSYN.
    \item Сброс SSYN в ответ на сброс MSYN.
\end{enumerate}
\end{block}
\end{frame}

\subsection{Арбитраж шины}

\begin{frame}
\frametitle{Централизованный арбитраж}
\screenshotw{11cm}{centr-arbitration.png}
\pause
\screenshotw{11cm}{centr-arbitration-levels.png}
\end{frame}

\subsection{Разные типы циклов работы шины}

\begin{frame}
\frametitle{Блочное чтение}
\screenshotw{11cm}{block-read-on-bus.png}
\end{frame}

\begin{frame}
\frametitle{Контроллер прерываний}
\screenshotw{12cm}{interrupt-controller.png}
\end{frame}

\section{Примеры шин}

\subsection{DDR}
\begin{frame}
\frametitle{Конвейерная работа с памятью DDR}
\begin{itemize}
    \item Матричная организация памяти: row x col.
    \pause\item Независимые «ранги» памяти (чипы, до 8 шт.) и «банки» (внутри одного чипа).
    \pause\item Основные команды для банка памяти:
    \begin{itemize}
        \item ACTIVATE — «открывает» $i$-ую строку;
        \item READ / WRITE — читает / пишет в $j$-й столбец;
        \item PRECHARGE — «закрывает» текущую открытую строку.
    \end{itemize}
\end{itemize}
\end{frame}

\begin{frame}
\frametitle{Пример: 12 циклов работы на шине памяти DDR3}
ACTIVATE = 2 такта, \hfill READ = 1 такт, \hfill PRECHARGE = 2 такта.
\screenshotw{12cm}{pipelined-memory.png}
\end{frame}

\subsection{USB}
\begin{frame}
\frametitle{Обзор}
Разработка: 1993–1998 г.\\
Compaq, DEC, IBM, Intel, Microsoft, NEC, и Northern Telecom
\begin{columns}
    \column{5cm}
\onslide<2->{\begin{block}{Задачи USB}
\begin{enumerate}
    \item «Нет» джамперам.
    \item «Нет» развинчиванию корпуса.
    \item Единый вид кабеля.
    \item Питание из кабеля.
    \item Поддержка устройств реального времени.
    \item «Горячая» установка.
    \item Дешевизна.
\end{enumerate}
\end{block}}

    \column{5cm}
    \screenshotw{5cm}{usb-logo.pdf}
\end{columns}
\end{frame}

\begin{frame}
\frametitle{Развитие шины USB}

\begin{columns}
    \column{5cm}

    \begin{block}{Версии шины}
    \begin{itemize}
        \item USB 1.0: 1.5 Mbps, 1996;
        \item USB 1.1: 12 Mbps, 1998;
        \item USB 2.0: 480 Mbps, 2000;
        \item USB 3.0: 5 Gbps, 2008.
    \end{itemize}
    \end{block}

    \column{6cm}
    \pause
    \begin{block}{Преимущества USB 3.0}
    \begin{itemize}
        \item Скорость,
        \item сила тока (900 вместо 500 мА),
        \item полный дуплекс,
        \item обратная совместмость.
    \end{itemize}
    \end{block}
\end{columns}

\pause

\begin{block}{Основные характеристики}
\begin{itemize}
    \item 4 канала: питание (5В), земля, данные (D+, D-);
    \item топология звезды (корневой хаб);
    \item сетевая архитектура (пакеты).
\end{itemize}
\end{block}
\end{frame}

\begin{frame}{Транзакции на шине USB}
\begin{columns}
    \pause
    \column{4.8cm}
    \begin{block}{Виды}
    \begin{itemize}
        \item Управляющие (control),
        \item прерывания (interrupt),
        \item пакетные (bulk),
        \item изохронные (isochronous).
    \end{itemize}
    \end{block}

    \pause
    \column{6.7cm}
    \centering
    Пример транзакций прерывания

    \screenshotw{6.7cm}{usb-interrupt.png}

    \screenshotw{3cm}{usb-transkey.png}
\end{columns}
\end{frame}

\begin{frame}{Примеры пакетов}%[fragile]

\begin{minipage}{7cm}
\begin{block}{Пакет IN: логическая структура}
\begin{table}[h]
\begin{tabular}{@{}cccccc@{}}
\toprule
Sync & PID & ADDR & ENDP & CRC5 & EOP\\\bottomrule
\end{tabular}
\end{table}
\end{block}
\end{minipage}

\vspace{.4cm}
\pause
\screenshotw{12cm}{USB_signal_example.pdf}
\end{frame}

\end{document}
