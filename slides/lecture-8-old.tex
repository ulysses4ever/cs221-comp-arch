\newcommand{\h}{%
handout,%
}

\documentclass[
\h%
ucs,ignorenonframetext,hyperref={pdftex,unicode},xcolor=dvipsnames]{beamer}
\usepackage[utf8x]{inputenc}
\usepackage[russian]{babel}
\usepackage[T2A]{fontenc}
\usepackage{tikz}
\usetikzlibrary{shapes,arrows,positioning,chains,calc}
\usepackage{graphicx}
\usepackage{fixltx2e}
\usepackage{paratype}
\usepackage{booktabs}
\usepackage{xcolor}
\usepackage{pbox}

\newcommand{\screenshot}[1]{
\begin{center}
\includegraphics[width=12cm,keepaspectratio]{./images/#1}
\end{center}
}

\newcommand{\screenshotw}[2]{
\begin{center}
\includegraphics[width=#1,keepaspectratio]{./images/#2}
\end{center}
}

\usecolortheme{crane}
\useoutertheme{infolines}
\setbeamertemplate{navigation symbols}{}

\author[А. М. Пеленицын]{А.~М.~Пеленицын\texorpdfstring{\\}{ }
apel@sfedu.ru}

\date{Весна 2016}

\institute[Мехмат ЮФУ]{Южный федеральный университет \texorpdfstring{\\}{ }
Институт математики, механики и компьютерных наук им. И.\,И.~Воровича\texorpdfstring{\\}{ }
Кафедра информатики и вычислительного эксперимента}

\subtitle{}

\AtBeginSection[]
{
\begin{frame}<beamer>
\frametitle{Содержание}
\tableofcontents[currentsection,hideothersubsections]
\end{frame}
}

\AtBeginSubsection[]
{
\begin{frame}<beamer>
\frametitle{Содержание}
\tableofcontents[currentsection,subsectionstyle=show/shaded/hide]
\end{frame}
}

\newcommand{\nspace}{\hspace{0pt}}
\newcommand{\nbdash}{\nobreakdash-\nspace}
\newcommand{\up}{\textsuperscript}
\newcommand{\bslash}{\textbackslash}

\newcommand{\link}[2]{\textcolor{blue}{\href{#1}{#2}}}

\newcommand{\Wrapped}[2][c]{%
  \begin{tabular}[#1]{@{}c@{}}#2\end{tabular}}


\usepackage{animate}

\title[Цифровая логика: вентили и схемы]{Архитектура компьютеров\texorpdfstring{\\}{ }Лекция 8. Цифровой логический уровень:\texorpdfstring{\\}{ }вентили и основные логические схемы}

\begin{document}

\begin{frame}
\titlepage
\end{frame}

\begin{frame}{Современные многоуровневые машины}
\screenshotw{6cm}{multi-level-computer.pdf}
\end{frame}

\begin{frame}
\frametitle{Содержание}
\tableofcontents
\end{frame}

\begin{frame}
\frametitle{Основные понятия цифрового логического уровня}
\Large\begin{itemize}
    \item Транзистор,
    \item вентиль (gate),
    \item интегральная схема.
\end{itemize}
\end{frame}

\section {Устройство и роль вентилей}
\begin{frame}
\frametitle{Использование транзисторов в реализации вентилей}
\begin{columns}
    \column{3cm} \pause\screenshotw{3.2cm}{transistor-NOT.png}
    \column{2.5cm} \pause\screenshotw{3.2cm}{transistor-NAND.png}
    \column{4.5cm} \pause\screenshotw{4.7cm}{transistor-NOR.png}
\end{columns}
\end{frame}

\begin{frame}
\frametitle{Обозначения для основных типов вентилей (ANSI)}
\screenshotw{7cm}{gates-2.png}
\pause
\screenshotw{10cm}{gates-1.png}
\end{frame}

\begin{frame}[plain]
\frametitle{Пример реализации функции 3-большинства}
\pause\begin{columns}
    \column{3cm} \screenshotw{3cm}{majority-table.png}
    \column{8cm}\pause \vspace{-.2cm}\screenshotw{8cm}{majority-curcuit.png}
\end{columns}
\end{frame}

\begin{frame}
\frametitle{Выбор базисных функций}

\vspace{1cm}\pause

\begin{columns}
    \column{5cm} Реализация NOT через\\NAND/NOR:\\
        \screenshotw{4cm}{NOT-via-NAND-NOR.png}
    \column{6cm} Реализация AND через NAND/NOR:\\
        \screenshotw{6cm}{AND-via-NAND-NOR.png}
\end{columns}

\pause\begin{itemize}
    \item Соображения эффективности ($AB + AC = A(B+C)$);
    \item предмет схемотехники;
    \item \link{http://clck.ru/99wJZ}{Википедия: Логические элементы}.
\end{itemize}
\end{frame}

\section {Базовые цифровые интегральные схемы}

\begin{frame}
\frametitle{Типы интегральных схем}
\screenshotw{6cm}{microchips.jpg}
\vspace{-.3cm}\pause\begin{itemize}
    \item Малые  интегральные (SSI, Small Scale Integrated) схемы: 1–10 вентилей,
    \item средние интегральные (MSI) схемы: 1–100 вентилей,
    \item большие интегральные (LSI) схемы: 100–100'000 вентилей,
    \item сверх большие интегральные (VLSI) схемы: > 100'000 вентилей.
\end{itemize}
\end{frame}


\begin{frame}
\frametitle{Пример МИС из четырёх вентилей}
\screenshotw{9cm}{ssi-circuit.png}
\pause\begin{itemize}
    \item \small Сколько нужно МИС, чтобы реализовать функцию 3-большинства?
    \item \small Сколько потребуется контактов для VLSI?
\end{itemize}
\end{frame}

\subsection{Корпусирование ИС}

\begin{frame}
\frametitle{Dual Inline Package, DIP / Quad Flat Package (QFP)}
Двустороннее / четырёхстороннее расположение выводов
\begin{columns}
    \column{6cm} \screenshotw{6cm}{DIP.jpg}
    \column{5cm} \screenshotw{5cm}{Sega_QFP.jpg}
\end{columns}
\end{frame}

\begin{frame}
\frametitle{Pin grid array, PGA}
Матрица штырьковых выводов
\begin{columns}
    \column{6cm} \screenshotw{6cm}{Pentium_4_PGA_Socket.jpg}
    \column{5cm} \screenshotw{5cm}{PGA_Socket_7.jpg}
\end{columns}
\end{frame}

\begin{frame}
\frametitle{Land grid array, LGA}
Матрица контактных площадок
\begin{columns}
    \column{6cm} \screenshotw{6cm}{Pentium_4_Prescott_LGA.jpg}
    \column{5.5cm} \screenshotw{5.5cm}{LGA_Socket.jpg}
\end{columns}
\end{frame}

\begin{frame}
\frametitle{Печатные платы (printed circuit board, PCB)}
\screenshotw{12cm}{PCB_design_and_realisation.png}
\end{frame}

\subsection{Комбинационые схемы}

\begin{frame}
\frametitle{Декодер}
\begin{itemize}
    \item $n$ входов, $2^n$ выходов
    \item пример: 3-декодер\\
    \vspace{-.1cm} \pause \screenshotw{7cm}{3-decoder.png}
\end{itemize}
\end{frame}

\begin{frame}
\frametitle{Мультиплексор}
\begin{itemize}
    \item $2^n$ входов, $n$ управляющих линий, один выход
    \item пример: 3-мультиплексор\\
    \pause \screenshotw{6cm}{3-multiplexer.png}
\end{itemize}
\end{frame}

\begin{frame}
\frametitle{Мультиплексор/демультиплексор: пример работы}
\centering
\animategraphics[width=10cm,loop,autoplay]{2}{images/mux/frame-}{0}{4}
\end{frame}

\begin{frame}
\frametitle{Пример: функция 3-большинства на основе 3-мультиплексора}
\screenshotw{5cm}{majority-via-multiplexer.png}
\end{frame}

\begin{frame}
\frametitle{Компаратор}
\begin{itemize}
    \item $2n$ входов, один выход
    \item \pause пример: 4-компаратор\\
    \screenshotw{6.6cm}{4-comparator.png}
\end{itemize}
\end{frame}

\begin{frame}[plain]
\frametitle{Программируемые логические матрицы}
\screenshotw{8cm}{pla.png}
\end{frame}

\subsection{Арифметические схемы}

\begin{frame}
\frametitle{Схема сдвига (shifter)}
\screenshotw{12cm}{shifter.png}
\end{frame}

\begin{frame}
\frametitle{Сумматоры}
\begin{columns}
    \column{5cm}Полусумматор:\\
    \onslide<2->{\screenshotw{5cm}{half-adder.png}}
    \column{6cm}Полный сумматор:\\
    \vspace{-.4cm}\onslide<3->{\screenshotw{5.5cm}{full-adder.png}}
\end{columns}
\end{frame}

\end{document}


\begin{frame}[plain]
\frametitle{Одноразрядная секция АЛУ}
\vspace{-1.0cm}
\begin{itemize}[<+->]
    \onslide<4->{\item $n$-разрядные сумматоры}
    \onslide<5->{\item параллельные сумматоры}
\end{itemize}

\vspace{-.3cm}\screenshotw{9.2cm}{1-ALU.png}
\end{frame}
