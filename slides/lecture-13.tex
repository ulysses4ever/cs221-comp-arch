\newcommand{\h}{%
handout,%
}

\documentclass[
\h%
ucs,ignorenonframetext,hyperref={pdftex,unicode},xcolor=dvipsnames]{beamer}
\usepackage[utf8x]{inputenc}
\usepackage[russian]{babel}
\usepackage[T2A]{fontenc}
\usepackage{tikz}
\usetikzlibrary{shapes,arrows,positioning,chains,calc}
\usepackage{graphicx}
\usepackage{fixltx2e}
\usepackage{paratype}
\usepackage{booktabs}
\usepackage{xcolor}
\usepackage{pbox}

\newcommand{\screenshot}[1]{
\begin{center}
\includegraphics[width=12cm,keepaspectratio]{./images/#1}
\end{center}
}

\newcommand{\screenshotw}[2]{
\begin{center}
\includegraphics[width=#1,keepaspectratio]{./images/#2}
\end{center}
}

\usecolortheme{crane}
\useoutertheme{infolines}
\setbeamertemplate{navigation symbols}{}

\author[А. М. Пеленицын]{А.~М.~Пеленицын\texorpdfstring{\\}{ }
apel@sfedu.ru}

\date{Весна 2016}

\institute[Мехмат ЮФУ]{Южный федеральный университет \texorpdfstring{\\}{ }
Институт математики, механики и компьютерных наук им. И.\,И.~Воровича\texorpdfstring{\\}{ }
Кафедра информатики и вычислительного эксперимента}

\subtitle{}

\AtBeginSection[]
{
\begin{frame}<beamer>
\frametitle{Содержание}
\tableofcontents[currentsection,hideothersubsections]
\end{frame}
}

\AtBeginSubsection[]
{
\begin{frame}<beamer>
\frametitle{Содержание}
\tableofcontents[currentsection,subsectionstyle=show/shaded/hide]
\end{frame}
}

\newcommand{\nspace}{\hspace{0pt}}
\newcommand{\nbdash}{\nobreakdash-\nspace}
\newcommand{\up}{\textsuperscript}
\newcommand{\bslash}{\textbackslash}

\newcommand{\link}[2]{\textcolor{blue}{\href{#1}{#2}}}

\newcommand{\Wrapped}[2][c]{%
  \begin{tabular}[#1]{@{}c@{}}#2\end{tabular}}


\AtBeginSection[]
{
\begin{frame}<beamer>
\frametitle{Содержание}
\tableofcontents[currentsection,hideallsubsections]
\end{frame}
}

\usepackage{animate}

\title[Цифровая логика: вентили и схемы]{Архитектура компьютеров\texorpdfstring{\\}{ }Лекция 13. Цифровой логический уровень:\texorpdfstring{\\}{ }вентили и основные логические схемы}

\begin{document}

\begin{frame}
\titlepage
\end{frame}

\begin{frame}{Современные многоуровневые машины}
\centering
{\small
\tikzset{
>=stealth',
  punktchain/.style={
    rectangle,
    rounded corners,
    % fill=black!10,
    draw=black, very thick,
    text width=15em,
    minimum height=2em,
    text centered,
    top color=white,
    on chain},
  every join/.style={->, thick,shorten >=1pt},
}
\begin{tikzpicture}
  [node distance=.7cm,
  start chain=going below,]
     \node[punktchain, join, bottom color=lime!60] (PL)
         {Языки «высокого уровня»};

     \node[punktchain, join, bottom color=teal!75] (ASM)
         {Язык ассемблера};

     \node[punktchain, join, bottom color=blue!50] (BIN)
         {«Машинный код»\\(уровень набора инструкций, Instruction Set Arch., \textbf{ISA})};

     \node[punktchain, join, bottom color=yellow!60] (MuCode)
         {Микрокод процессора\\(\textbf{микроархитектура})};

     \node[punktchain, join, bottom color=red!60] (DL)
         {Схемы цифровой логики};
\end{tikzpicture}
}

\end{frame}

\begin{frame}
\frametitle{Содержание}
\tableofcontents
\end{frame}

\begin{frame}
\frametitle{Основные понятия цифрового логического уровня}
\begin{itemize}\itemsep=.5cm
    \item Транзистор,
    \item \Large вентиль (gate),
    \item \LARGE  интегральная схема.
\end{itemize}
\end{frame}

\section {Устройство и роль вентилей}
\begin{frame}
\frametitle{Использование транзисторов в реализации вентилей}
\begin{columns}
    \column{3cm} \pause\screenshotw{3.2cm}{transistor-NOT.png}
    \column{2.5cm} \pause\screenshotw{3.2cm}{transistor-NAND.png}
    \column{4.5cm} \pause\screenshotw{4.7cm}{transistor-NOR.png}
\end{columns}

\pause Идея множественных входов.
\end{frame}

\begin{frame}
\frametitle{Обозначения для основных типов вентилей (ANSI)}
\screenshotw{7cm}{gates-2.png}
\pause
\screenshotw{10cm}{gates-1.png}
\end{frame}

\begin{frame}[plain]
\frametitle{Пример реализации функции 3-большинства}
\pause\begin{columns}
    \column{3.5cm} \screenshotw{3cm}{majority-table.png}
    \pause СДНФ:\\\small$\bar A BC + A \bar B C + AB \bar C + ABC$
    \column{8cm}\pause \vspace{-.2cm}\screenshotw{8cm}{majority-curcuit.png}
\end{columns}
\end{frame}

\begin{frame}
\frametitle{Выбор базисных функций}

\vspace{1cm}\pause

\begin{columns}
    \column{5cm} Реализация NOT через\\NAND/NOR:\\
        \screenshotw{4cm}{NOT-via-NAND-NOR.png}
    \column{6cm} Реализация AND через NAND/NOR:\\
        \screenshotw{6cm}{AND-via-NAND-NOR.png}
\end{columns}

\pause\begin{itemize}
    \item Соображения эффективности ($AB + AC = A(B+C)$);
    \item предмет схемотехники;
    \item \link{http://clck.ru/99wJZ}{Википедия: Логические элементы}.
\end{itemize}
\end{frame}

\section {Базовые цифровые интегральные схемы}

\begin{frame}{Что нам потребовалось для реализации Mic-1}

\begin{columns}
  \column{6cm}
  \begin{block}{Комбинационные схемы}
  \begin{itemize}
    \item АЛУ,
    \item схема сдвига,
    \item декодер.
  \end{itemize}
  \end{block}

  \column{6cm}\pause
  \begin{block}{Схемы памяти («последовательностные»)}
  \begin{itemize}
    \item Регистр (параллельный, синхронный),
    \item память на ~2KB,
    \item регистр сдвига (Mic-2).
  \end{itemize}
  \end{block}
\end{columns}

\pause
\begin{block}{Тактовые генераторы}
\end{block}
\end{frame}

\subsection{Комбинационые схемы}

\begin{frame}
\frametitle{Декодер}
\begin{itemize}
    \item $n$ входов, $2^n$ выходов
    \item пример: 3-декодер\\
    \vspace{-.1cm} \pause \screenshotw{7cm}{3-decoder.png}
\end{itemize}
\end{frame}

\begin{frame}
\frametitle{Мультиплексор}
\begin{itemize}
    \item $2^n$ входов, $n$ управляющих линий, один выход.\pause

    \item Пример: 2-мультиплексор\\
    \screenshotw{8cm}{2-mux.jpg}
\end{itemize}
\end{frame}

\begin{frame}{Мультиплексор через декодер}\pause
\screenshotw{8cm}{2-mux-via-decoder.png}
\end{frame}

\begin{frame}
\frametitle{Мультиплексор/демультиплексор: пример работы}
\centering
\animategraphics[width=12cm,loop,autoplay]{2}{images/mux/frame-}{0}{4}
\end{frame}

\begin{frame}
\frametitle{Пример: функция 3-большинства на основе 3-мультиплексора}
\screenshotw{5cm}{majority-via-multiplexer.png}
\end{frame}

\begin{frame}
\frametitle{Компаратор}
\begin{itemize}
    \item $2n$ входов, один выход
    \item \pause пример: 4-компаратор\\
    \screenshotw{6.6cm}{4-comparator.png}
\end{itemize}
\end{frame}

\begin{frame}[plain]
\frametitle{Программируемые логические матрицы, ПЛИС (FPGA)}
\vspace{-.3cm}
\screenshotw{8.5cm}{pla.png}
\end{frame}

\subsection{Арифметические схемы}

\begin{frame}
\frametitle{Схема сдвига (shifter)}
\screenshotw{12cm}{shifter.png}
\end{frame}

\begin{frame}
\frametitle{Сумматоры}
%\begin{columns}
%    \column{5.5cm}
%    \pause
\begin{minipage}{5.5cm}
    \vspace{-2cm}
    Как сложить два бита?
    \pause\screenshotw{5cm}{half-adder.png}

    \vspace{-2.5cm}

    \pause \hfill — полусумматор
\end{minipage}
\hfill
%
%
%    \column{5.5cm}
%    \hspace{0cm}
%
%    \vspace{1cm}
%
\begin{minipage}{6cm}
    \vspace{1cm}
    \pause Полный сумматор:\\
    \vspace{-1.5cm}
    \screenshotw{5.5cm}{full-adder-a.png}
\end{minipage}
%\end{columns}

\vspace{-1cm}

\begin{itemize}[<+->]
    \onslide<5->{\item $n$-разрядные сумматоры;}
    \onslide<6->{\item параллельные сумматоры.}
\end{itemize}
\end{frame}

\begin{frame}{Идея организации АЛУ (1-разрядная секция)}
\screenshotw{7.5cm}{1-ALU-abstract.png}
\end{frame}

\begin{frame}[plain]
\frametitle{Одноразрядная секция АЛУ для Mic-1}
\begin{columns}

        \column{3.5cm}\onslide<2>{
    Номера команд:
    \begin{itemize}
        \item 0 — A \textbf{AND} B
        \item 1 — \textbf{NOT} B
        \item 2 — A \textbf{OR} B
        \item 3 — A \textbf{+} B
    \end{itemize}}

        \column{9cm}
    \vspace{-.3cm}
    \screenshotw{9cm}{1-ALU.png}
\end{columns}
\end{frame}


\begin{frame}
\frametitle{Полезные сочетания управляющих входов АЛУ}
\begin{columns}

        \column{3.5cm}
    Номера команд:
    \begin{itemize}
        \item 0 — A \textbf{AND} B
        \item 1 — \textbf{NOT} B
        \item 2 — A \textbf{OR} B
        \item 3 — A \textbf{+} B
    \end{itemize}

        %\pause
        \column{9cm}
    \vspace{-.3cm}\screenshotw{7cm}{alu-table.png}
\end{columns}
\end{frame}

\end{document}

\section{Корпусирование и типы ИС}

\begin{frame}
\frametitle{Типы интегральных схем}
\screenshotw{6cm}{microchips.jpg}
\vspace{-.3cm}\pause\begin{itemize}
    \item Малые  интегральные (SSI, Small Scale Integrated) схемы: 1–10 вентилей,
    \item средние интегральные (MSI) схемы: 1–100 вентилей,
    \item большие интегральные (LSI) схемы: 100–100'000 вентилей,
    \item сверх большие интегральные (VLSI) схемы: > 100'000 вентилей.
\end{itemize}
\end{frame}


\begin{frame}
\frametitle{Пример МИС из четырёх вентилей}
\screenshotw{9cm}{ssi-circuit.png}
\pause\begin{itemize}
    \item \small Сколько нужно МИС, чтобы реализовать функцию 3-большинства?
    \item \small Сколько потребуется контактов для VLSI?
\end{itemize}
\end{frame}

\begin{frame}
\frametitle{Dual Inline Package, DIP / Quad Flat Package (QFP)}
Двустороннее / четырёхстороннее расположение выводов
\begin{columns}
    \column{6cm} \screenshotw{6cm}{DIP.jpg}
    \column{5cm} \screenshotw{5cm}{Sega_QFP.jpg}
\end{columns}
\end{frame}

\begin{frame}
\frametitle{Pin grid array, PGA}
Матрица штырьковых контактов
\begin{columns}
    \column{6cm} \screenshotw{6cm}{Pentium_4_PGA_Socket.jpg}
    \column{5cm} \screenshotw{5cm}{PGA_Socket_7.jpg}
\end{columns}
\end{frame}

\begin{frame}
\frametitle{Land grid array, LGA}
Матрица контактных площадок
\begin{columns}
    \column{6cm} \screenshotw{6cm}{Pentium_4_Prescott_LGA.jpg}
    \column{5.5cm} \screenshotw{5.5cm}{LGA_Socket.jpg}
\end{columns}
\end{frame}

\begin{frame}
\frametitle{Печатные платы (printed circuit board, PCB)}
\screenshotw{12cm}{PCB_design_and_realisation.png}
\end{frame}

