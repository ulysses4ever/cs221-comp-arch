\newcommand{\h}{%
handout,%
}

\documentclass[
\h%
ucs,ignorenonframetext,hyperref={pdftex,unicode},xcolor=dvipsnames]{beamer}
\usepackage[utf8x]{inputenc}
\usepackage[russian]{babel}
\usepackage[T2A]{fontenc}
\usepackage{tikz}
\usetikzlibrary{calc,trees,positioning,arrows,chains,shapes.geometric,%
    decorations.pathreplacing,decorations.pathmorphing,shapes,%
    matrix,shapes.symbols}
\usepackage{graphicx}
\usepackage{fixltx2e}
\usepackage{paratype}
\usepackage{booktabs}
\usepackage{xcolor}
\usepackage{pbox}

\usepackage[font=itshape]{quoting}

\newcommand{\screenshot}[1]{
\begin{center}
\includegraphics[width=12cm,keepaspectratio]{./images/#1}
\end{center}
}

\newcommand{\screenshotw}[2]{
\begin{center}
\includegraphics[width=#1,keepaspectratio]{./images/#2}
\end{center}
}

\usecolortheme{crane}
\useoutertheme{infolines}
\setbeamertemplate{navigation symbols}{}

\author[А. М. Пеленицын]{А.~М.~Пеленицын\texorpdfstring{\\}{ }
apel@sfedu.ru}

\date{Весна 2016}

\institute[Мехмат ЮФУ]{Южный федеральный университет \texorpdfstring{\\}{ }
Институт математики, механики и компьютерных наук им. И.\,И.~Воровича\texorpdfstring{\\}{ }
Кафедра информатики и вычислительного эксперимента}

\subtitle{}

\AtBeginSection[]
{
\begin{frame}<beamer>
\frametitle{Содержание}
\tableofcontents[currentsection,hideothersubsections]
\end{frame}
}

\AtBeginSubsection[]
{
\begin{frame}<beamer>
\frametitle{Содержание}
\tableofcontents[currentsection,subsectionstyle=show/shaded/hide]
\end{frame}
}

\newcommand{\nspace}{\hspace{0pt}}
\newcommand{\nbdash}{\nobreakdash-\nspace}
\newcommand{\up}{\textsuperscript}
\newcommand{\bslash}{\textbackslash}

\newcommand{\link}[2]{\textcolor{blue}{\href{#1}{#2}}}

\newcommand{\Wrapped}[2][c]{%
  \begin{tabular}[#1]{@{}c@{}}#2\end{tabular}}

\setbeamertemplate{enumerate subitem}{(\roman{enumii})}


\usepackage{multicol}

\title[Архитектура компьютеров. Заключение]{Архитектура компьютеров. Заключение}


\newcommand{\LRA}{$\leftrightarrow$}

\AtBeginSection[]
{
  \begin{frame}<beamer>
  \small
   \begin{multicols}{2}
     \tableofcontents[currentsection,hideothersubsections]
   \end{multicols}
  \end{frame}
}

%%%%%%%%%%%%%%%%%%%%%%%%%%%% НАЧАЛО ДОКУМЕНТА %%%%%%%%%%%%%%%%%%%%%%%%%%%%
\begin{document}

\begin{frame}
\titlepage
\end{frame}

\begin{frame}{Содержание курса}
\small
\begin{multicols}{2}
  \tableofcontents
\end{multicols}
%\small\tableofcontents%[hideallsubsections]
\end{frame}

%\screenshotw{3.2cm}{transistor-NOT.png}

\section{Лекция 1. Введение}
\begin{frame}{Организация и архитектура}
\begin{columns}
	\column{6cm}
	\screenshotw{6cm}{a-computer.pdf}

	\column{6cm}
\screenshotw{6cm}{multi-level-computer.pdf}
\end{columns}
\end{frame}

\section{Лекция 2. Развитие вычислительной техники}


\begin{frame}{Поколения компьютерной техники}
\begin{itemize}
    \item Нулевое поколение — механические компьютеры (1642–1945)
    \item Первое поколение — электронные лампы (1945–1955)
    \item Второе поколение — транзисторы (1955–1965)
    \item Третье поколение — интегральные схемы (1965–1980)
    \item Четвёртое поколение — СБИС (VLSI), микропроцессоры, персональные компьютеры (1980–?)
    \item Пятое поколение — невидимые компьютеры
\end{itemize}
\end{frame}

\begin{frame}{Первое поколение}
\screenshotw{12cm}{history-1.png}
\end{frame}

\begin{frame}{Транзисторы (1947 г.): Нобелевская премия 1956 г.}
\screenshotw{8.5cm}{Bardeen_Shockley_Brattain_1948.JPG}
\vspace{-.3cm}\small Джон Бардин, Уильям Шокли и Уолтер Браттейн в лаборатории Bell, 1948 г.
\end{frame}

\section[{Лекция 3. Разнообразие компьютеров}]{Лекция 3. Разнообразие компьютеров}

\begin{frame}[plain]{Закон Мура (Гордон Мур, Дэвид Хаус, 1965)}
\vspace{-.2cm}\screenshotw{11cm}{moores_law2.pdf}
\end{frame}

\begin{frame}{Категории вычислительных устройств}
\begin{itemize}
    \item «Одноразовые компьютеры»,
    \item микроконтроллеры (встраиваемые компьютеры),
    \item игровые приставки,
    \item персональные компьютеры,
    \item серверы,
    \item высокопроизводительные вычисления (HPC).
\end{itemize}
\end{frame}

%Современные процессоры и параллелизм. Оперативная память.
\section{Лекция 4. Центральный процессор}
\begin{frame}{Схема работы CPU}
\color{red}{Красным} \color{black}{— поток управления, чёрным — поток данных.}
\screenshotw{10cm}{cpu.png}
\end{frame}

\begin{frame}
\frametitle{Конвейеры и суперскаляры, таксономия Флинна}
\begin{columns}
    \column{5cm}
\vspace{-1.8cm}
\screenshotw{7cm}{pipeline-1.png}
\vspace{-.5cm}
\screenshotw{5cm}{pipeline-2.png}
    \column{6cm}
\vspace{.2cm}
\screenshotw{6.5cm}{superscalar.png}
\end{columns}
\begin{itemize}
    \item SISD,
    \item SIMD,
    \item MISD,
    \item MIMD.
\end{itemize}
\end{frame}

\section{Лекция 5. Оперативная память}
\begin{frame}{Логическая структура: порядок байт}
\screenshotw{11cm}{big-little.png}
\end{frame}

\section[Вторичная память]{Лекции 6, 7.  Вторичная память}

\begin{frame}
\frametitle{Иерархия памяти}
\screenshotw{10cm}{memory-hierarchy.png}
\end{frame}

\begin{frame}{Помехоустойчивое кодирование}
\begin{columns}
    \column{7cm}
\begin{block}{$(7, 4)$-код Хэмминга}
        \screenshotw{6cm}{Hamming_7_4.pdf}
\end{block}
    \column{0cm}
\end{columns}
\end{frame}

\section[Подсистема ввода-вывода]{Лекции 6, 7.  Подсистема ввода-вывода}
\begin{frame}{Прерывания, их источники}
        \screenshotw{6.8cm}{interrupt-process.png}
\end{frame}

\section[Вентили и основные логические схемы]{Лекции 8–10.  Уровень цифровой логики: вентили, схемы, шины}
\begin{frame}
\frametitle{Типы логических схем и соединений}
\begin{itemize}
    \item Комбинационные,
    \item последовательностные (sequential, схемы памяти),
    \item шины,
    \begin{itemize}
        \item внешние и внутренние,
        \item параллельные и последовательные.
    \end{itemize}
\end{itemize}
\end{frame}

\begin{frame}[plain]
\frametitle{Организация типичного ПК 2010-х}
\screenshotw{10.5cm}{h87.png}
\end{frame}


\section{Лекция 11–12.  Уровень микроархитектуры}
\begin{frame}
\frametitle{Пример микроархитектуры: Mic-1}
\vspace{-.3cm}
\screenshotw{8cm}{Mic-1.png}
\end{frame}

\begin{frame}[plain]{Микроархитектура Intel Core 2 (2006)}
\screenshotw{7cm}{Intel_Core2_muarch.pdf}
\end{frame}

\section[Уровень набора инструкций]{Лекция 13. Уровень набора инструкций на примере MIPS}
\begin{frame}
\begin{block}{Определение из Википедии}
An \textbf{instruction set}, or \textbf{instruction set architecture (ISA)}, is the part of the computer architecture \alert{related to programming}, including the
\begin{columns}
    \column{6cm}
\begin{itemize}
    \item native data types,
    \item instructions,
    \item registers,
    \item addressing modes,
\end{itemize}
    \column{6cm}
\begin{itemize}
    \item memory architecture,
    \item interrupt and exception handling,
    \item external I/O.
\end{itemize}
\end{columns}
\end{block}

\pause CISC (x86, IBM z Series) против RISC (MIPS, ARM, …).
\end{frame}

\section{Лекция 14. Числовые типы данных}
\begin{frame}
\frametitle{Типы чисел в IEEE 754}
\screenshotw{12cm}{ieee-754-number-types.png}
\end{frame}

\section{Заключение}

\begin{frame}
\frametitle{Чему нужно учить в курсе архитектуры компьютера}
\pause
\begin{columns}
    \column{0cm}
    \column{10cm}
\begin{block}{Чему нужно учить специалистов по информатике}
    \textcolor{blue}{\href{https://www.acm.org/education/CS2013-final-report.pdf}{Computer Science Curricula 2013}}
    \small by
    \begin{itemize}
        \item Association for Computing Machinery (ACM),
        \item IEEE Computer Society.
    \end{itemize}
\end{block}
\end{columns}

\pause
\begin{block}{Фрагмент CSC2013: Architecture and Organization, цитата}
A professional in any field of computing should \textbf{not} regard the computer as just a \emph{black
box} that executes programs by \emph{magic}.
\end{block}
\end{frame}

\begin{frame}{CSC 2008: AR/Directions in Computing [elective]}
\begin{itemize}
    \item Semiconductor technology and Moore’s law,
    \item limitations to semiconductor technology,
    \item quantum computing,
    \item optical computing,
    \item molecular (biological) computing,
    \item new memory technologies.
\end{itemize}
\end{frame}

\end{document}
